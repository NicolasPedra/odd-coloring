\documentclass[12pt]{article}
\usepackage{cancel}
\usepackage{sbc-template/sbc-template}
\usepackage{graphicx,url}
\usepackage[utf8]{inputenc}
\usepackage[brazil]{babel}    
\usepackage{geometry}
\usepackage{setspace} 
\usepackage[T1]{fontenc}
\usepackage[brazil]{babel}
\usepackage{amsthm}
\usepackage[portuguese, ruled, lined, linesnumbered, commentsnumbered, longend]{algorithm2e}
\usepackage{blindtext}
\usepackage{geometry}
\usepackage{scrextend}
\usepackage{setspace}
\usepackage{amssymb}
\usepackage{amsmath}
\usepackage{dsfont}
 
\newcommand{\mycomment}[1]{}
\newenvironment{citacao}{%
\begin{list}{}{%
\setlength{\topsep}{1pt}%
\setlength{\leftmargin}{4cm}%
\setlength{\rightmargin}{0cm}%
\setlength{\listparindent}{\parindent}%
\setlength{\itemindent}{\parindent}%
\setlength{\parsep}{\parskip}%
}%
\item[]}{\end{list}}

%\renewcommand{\baselinestretch}{0.6} 

\sloppy
\raggedbottom
 
\begin{document} 
 \def\changemargin#1#2{\list{}{\rightmargin#2\leftmargin#1}\item[]} 
\let\endchangemargin=\endlist 
\newcommand{\wbigcup}{\mathop{\widetilde{\bigcup}}\displaylimits}

\newcommand{\ie}{\textit{i}.\textit{e}.} 
\newcommand{\blceil}{\left \lceil}
\newcommand{\brceil}{\right \rceil} 
\newcommand{\blfloor}{\left \lfloor}
\newcommand{\brfloor}{\right \rfloor} 
\newcommand{\defcomunidade}{Def.1 \ (a) \text{,} \ (b) \ \text{e} \ (c)}

\newcommand{\newbegin}{\vspace{0.5cm}}

\newcommand{\newl}{\vspace{0.1cm}}
\newcommand{\escolhe}[2]{\resizebox{!}{15pt}{$\displaystyle \binom{#1}{#2}$}}

\newtheorem{definicao}{Def} 
\newtheorem{exemplo}{Exemplo}  
\newtheorem{lema}{Lema} 
\newtheorem{teo}{Teorema} 
\newtheorem{cor}{Corolário}
\newtheorem{prop}{Proposição}
\newtheorem{defi}{Definição}

%LEMA 
%LEMA 
%LEMA 
%LEMA


%LEMA 1 
%LEMA 1 
%LEMA 1 
%LEMA 1  
\begin{prop}
	\label{prop1}
	Seja $G$ um grafo conexo. Seja $T$ uma árvore de Busca em Largura de $G$ a partir de um vértice $v$ qualquer. Se existem vértices $s, t \in V(G)$ tais que $st \in E(G)$, $st \notin E(T)$ e $dist_T(v, s) = dist_T(v, t) + 1$, então $G$ possui ciclo par.
\end{prop} \newbegin

\begin{prop}
	\label{prop2}
	Seja $G$ um grafo conexo livre de ciclos pares. Seja $T$ uma Árvore de Busca em Largura de $G$ a partir de $v \in V(G)$. Seja $V \subsetneq V(G)$ o conjunto de vértices com distância $p > 0$ de $v$. Temos que o conjunto $E(G[V])$ é um emparelhamento.
\end{prop} \newbegin


\begin{lema}  
\label{lema1}
Seja $G$ um grafo. Se existe uma $k$-partição $V_1, V_2, \ldots ,V_k$ dos vértices de $G$ tal que, para todo vértice $v \in V(G)$, temos que $|N(v) \cap V_i| = 1$, para algum $1 \leq i \leq k$, então $\chi_{pcf}(G) \leq  \sum\limits_{i = 1}^{k} \chi(G[V_i])$.
\end{lema}

\begin{proof}
	
 Seja $H_i = G[V_i]$, para todo $1 \leq i \leq k$. Iremos colorir cada subgrafo $H_i$ com $\chi(H_i)$ cores distintas. Para isso, cada cor será representada por um par ordenado. Seja $c_i: V(H_i) \rightarrow \{ i \} \times \chi(H_i)$ uma coloração própria de $H_i$. Para todo par distinto de colorações $c_i$ e $c_j$, temos que $c_i(v) \neq c_j(u)$, para todo $ v \in V(H_i)$ e $ u \in V(H_j)$, pois $(i, x) \neq (j, y)$ para $i \neq j$.\newl
  
  Seja $c$ uma coloração de $G$ tal que $c(v) = c_i(v)$ se e somente se $v \in V(H_i)$. Em outras palavras, $c$ é a união das colorações usadas em cada subgrafo $H_i$. Como $c_i$ é uma coloração própria de $H_i$ e todo par distinto de subgrafos $H_i$ e $H_j$ são coloridos com cores distintas, temos que $c$ é uma coloração própria de $G$. \newl
 
  Como, para todo vértice $v \in V(G)$, vale que $|N(v) \cap V(H_i)| = 1$, para algum subgrafo $H_i$, e como as cores usadas em $H_i$ são distintas das cores usadas em $V(G)\setminus V(H_i)$, temos que existe uma cor $(i, x)$ que aparece uma única vez na vizinhança de $v$, para $x \in [\chi(H_i)]$. Sendo assim, $c$ descreve uma coloração própria livre de conflitos de $G$. Como $c_i$ utiliza $\chi(H_i)$ cores, para todo $1 \leq i \leq k$, temos que $\chi_{pcf}(G) \leq \sum\limits_{i = 1}^{k} \chi(H_i)$. 
  
\end{proof} 


 \begin{teo}
 \label{teo1}
 	Seja $G$ um grafo conexo. Se $G$ é livre de ciclos pares, então $\chi_{pcf}(G) \leq 7$.
 \end{teo}
 
 \begin{proof}
 	Seja $T$ uma Árvore de Busca em Largura de $G$ a partir de um vértice $r$ qualquer. Sabemos que $T$ é uma árvore geradora, pois $G$ é conexo. Seja $V_0, V_1$, $V_2$ uma partição de $G$ tal que $x \in V_i$ se e somente se $i = dist_T(r, x) \pmod{3} $. Seja $s$ um vértice de $G$, tal que $s \neq r$. Seja $p$ o pai de $s$ em $T$ e seja $f$ um filho de $s$ em $T$.
 	Note que $p$ e $f$ pertencem a partições distintas, pois:
 	\begin{align}
 		\begin{split}
 			dist_T(r, p) \pmod{3} \neq dist_T(r, p) + 2 \pmod{3} = dist_T(r, f) \pmod{3}
 		\end{split} 
 	\end{align} 
   
 	 Seja $t \in V(G)$ um vértice tal que $dist_T(r, s) > dist_T(r, t) + 1$. Sabemos que $st \notin E(G)$, pois $T$ é uma árvore de Busca em Largura. Sendo assim, se $st \in E(G)$ e $st \notin E(T)$, então $dist_T(r, s) = dist_T(r, t) + 1$ ou $dist_T(r, s) = dist_T(r, t)$. Pela Proposição $\ref{prop1}$, sabemos que se $st \in E(G)$ e $st \notin E(T)$, então $dist_T(r, s) = dist_T(r, t)$, pois $G$ é livre de ciclos pares. Note que isto implica que $s$ é adjacente a precisamente um vértice $u$ em $G$ tal que $dist_T(r, s) = dist_T(r, u) + 1$, e, sendo assim, $u$ é o pai de $s$ em $T$, \ie, $u = p$. Note que $|N(s) \cap V_i| = 1$, onde $f \in V_i$, para todo $s \in V(G) \setminus\{r\}$. \newl
 	 
 	 Resta agora a partição do vértice raiz $r$. Iremos remover um vértice $v \in N(r)$ da partição $V_1$ e iremos construir uma nova partição $V_0, V^{'}_1, V_2, V_3$ de $G$, de modo que $V^{'}_1 = V_1\setminus\{v\}$ e $V_3 = \{v\}$. Seja $s$ um vértice onde $|N(s) \cap V_1| = 1$, \ie, o pai $p$ de $s$ pertence a $V_1$. Queremos argumentar que a propriedade é satisfeita para $s$ na nova partição $V_0, V^{'}_1, V_2, V_3$. Se $p \neq v$, então $|N(s) \cap V_1'| = 1$ e a propriedade continua valendo. Se $p = v$, então $N(s) \cap V_3 = \{v\}$, \ie, $ |N(s) \cap V_3| = 1$ e a propriedade vale. \newl
 	 
 	 Note que a partição $V_0, V_1', V_2 \text{ e } V_3$ satisfaz a condição do Lema $\ref{lema1}$. Note que pela Proposição $\ref{prop2}$, $E(G[V_i])$ é um emparelhamento. Sendo assim, temos que $\chi(G[V_i]) = 2$, para $0 \leq i \leq 2$. Note que $\chi(G[V_3]) = 1$, pois $V_3 = \{v\}$. Sendo assim, pelo Lema $\ref{lema1}$, temos que $\chi_{pcf}(G) \leq 7$.
  
 \end{proof}  \newpage
 
 \begin{defi}
 	\label{defi1}
 	Seja $\mathcal{P} = \{P_1, P_2, \ldots P_k \}$ uma $k-$partição de $d$ elementos distintos. Dizemos que $\mathcal{P}$ é uma $k-$partição par se $|P_i|$ é par, para todo $P_i \in \mathcal{P}$.
 \end{defi} \newbegin
  
 
 \begin{defi}
 	\label{defi2}
 	Denotamos por $\mu(d, k)$ a quantidade de $k-$partições pares distintas de $d$ elementos distintos.
 \end{defi} \newbegin
 
  \begin{defi}
 	\label{defi3}
 	Denotamos por $\Phi(d, k)$ a quantidade de $k-$partições não pares distintas de $d$ elementos distintos. Denotamos por $\Phi_\ell(d, k)$ a quantidade de $k-$partições não pares de $d$ elementos, com exatamente $\ell$
 	partes de cardinalidade ímpar. Também denotamos por $\Phi_{\geq \ell}(d, k)$ a quantidade de $k-$partições não pares de $d$ elementos com pelo menos $\ell$ partes de cardinalidade ímpar, \ie, $\Phi_{\geq \ell}(d, k) = \sum\limits_{i = \ell}^{k} \Phi_{i}(d, k)$. 
 \end{defi} \newbegin
  
   \begin{defi}
 	\label{def4}
   Denotamos por $\varphi_\ell(d, k)$ a quantidade de $k-$partições não pares de $d$ elementos onde somente as primeiras $\ell$ partes tenham cardinalidade ímpar. Claramente temos que $\varphi_\ell(d, k) \leq \Phi_\ell(d, k)$, pois $\varphi_\ell(d, k)$ conta apenas as $k-$partições com as primeiras $\ell$ partes de cardinalidade ímpar, já $\Phi_\ell(d, k)$ conta qualquer subconjunto pertencente a $\escolhe{[k]}{\ell}$ com cardinalidade ímpar.
 \end{defi} \newbegin
 
  \begin{defi}
 	\label{def5}
 	Denotamos por $\chi_{io}(G)$ o menor inteiro $k$ tal que  $G$ possui uma $k-$coloração ímpar não própria. 
 \end{defi} \newbegin
 
 \begin{prop}
 	\label{prop3}
 	Seja $2n!!$ o fatorial dos ímpares. Temos que $2n!! \leq n^n$.
 \end{prop} \newbegin
 
 
 \begin{lema}  
 	\label{lema2}
 	Seja a recorrência a seguir: \\
 	
 		\begin{equation}
 		T(2d, k) =
 		\begin{cases}
 			1 & \text{$k = 1$} \\
 			\sum\limits_{i = 0}^{dn} \escolhe{2d}{2i} \cdot T(2i, k-1) & \text{c.c.} \\ 
 		\end{cases}
 	\end{equation}
 	
 	Temos que $T(2d, k) = \mu(2d, k)$, para $k \geq 1$.
 \end{lema}
 
 \begin{proof}
 	A demonstração segue por indução em $k$. \newl
 	
 	Base ($k = 1$): Para $k = 1$, temos que $T(2d, k) = \mu(2d, 1) = 1$, pois há uma única partição $\mathcal{P}=\{P_1\}$ de $2d$ elementos, de modo que $|P_1|$ seja par. Sendo assim, o resultado segue. \newl
 	
 	Passo ($k > 1$): Suponha que $T(2d, \ell) = \mu(2d, \ell)$, para todo $1 \leq \ell < k$. Como $P_k$ tem tamanho par, temos que existe um $i$ tal que $|P_k| = 2i$, onde $0 \leq i \leq d$. Sabemos que há $\escolhe{2d}{2i}$ maneiras de escolher $2i$ elementos de $2d$ elementos para a parte $P_k$. Note que ao escolher $2i$ elementos para a parte $P_k$ temos que particionar os $2d - 2i$ elementos restantes em $(k-1)$ partes, de modo que cada parte tenha tamanho par. Sendo assim, temos que escolher uma $(k-1)-$partição par $\mathcal{P'}$ de $2d - 2i$ elementos. Note que há $\mu(2d - 2i, k-1)$ maneiras de escolher $\mathcal{P'}$. Sendo assim: 
 	
 	\begin{align}
 		\begin{split}
 			\mu(2n, k) = \sum\limits_{i = 0}^{d}\escolhe{2d}{2i} \cdot \mu(2d - 2i, k-1)
 		\end{split} 
 	\end{align} 
 	
 	Por $HI$, $T(2d - 2i, k-1) = \mu(2d - 2i, k-1)$. Note que $\escolhe{2d}{2i} = \escolhe{2d}{2d - 2i} $. Logo:
 	
 	\begin{align}
 		\begin{split}
 			\mu(2d, k) &= \sum\limits_{i = 0}^{n}\escolhe{2d}{2d - 2i} \cdot T(2d - 2i, k-1)\\
 			&= \sum\limits_{i = 0}^{d}\escolhe{2d}{2i} \cdot T(2i, k-1) = T(2d, k)
 		\end{split} 
 	\end{align}
 	
 \end{proof} \newl
 
 
  \begin{lema}  
 	\label{lema3}
 	Seja a recorrência a seguir: \\
 	
 	\begin{equation}
 		R(2d, k) =
 		\begin{cases}
 			0 & \text{se $k \leq 1$}\\
 			2^{2d - 1} & \text{se $k = 2$} \\
 			\sum\limits_{i = 1}^{d} \escolhe{2d}{2i} \cdot R(2i, k-1) & \text{se $k > 2$} \\ 
 		\end{cases}
 	\end{equation}
 	
 	Temos que $R(2d, k) = \varphi_2(2d, k)$.
 \end{lema}
  
  \begin{proof}
  	A demonstração segue por indução em $k$.
  	
  	Base ($k = 2$): Note que $\varphi_2(2d, 2) = \Phi(2d, 2)$, pois, seja uma $k-$partição não par $\mathcal{P}=(P_1, P_2)$, como $2d$ é par, temos que $|P_1| \text{ e } |P_2|$ são pares ou $|P_1| \text{ e } |P_2|$ são ímpares. Note que $\Phi(2d, 2) = 2^{2d} - \mu(2d, 2)$, pois há exatamente $2^{2d}$ formas de particionar $2d$ elementos em duas partes $P_1$ e $P_2$ e dessas $2^{2d}$ maneiras há $\mu(2d, 2)$ maneiras de particionar $2d$ termos tal que $|P_1|$ e $|P_2|$ sejam pares. Note que:  
  	
  	\begin{align}
  		\begin{split}
  			\mu(2d, 2) &= \sum\limits_{i = 0}^{d}\escolhe{2d}{2i} \cdot T(2i, 1) = \sum\limits_{i = 0}^{d}\escolhe{2d}{2i} = 2^{2d-1}
  		\end{split} 
  	\end{align} 
  	
  	Logo, temos que $\varphi_2(2d, 2) = \Phi(2d, 2) = 2^{2d} - 2^{2d - 1} = 2^{2d - 1} = R(2d, 2)$ e o resultado segue. \newl
  	
  	
  	Passo ($k > 1$): Suponha que $R(2d, \ell) = \varphi_2(2d, \ell)$, para todo $2 \leq \ell < k$.
  	 
  	
  	 Como somente as partes $P_1$ e $P_2$ tem tamanho ímpar, temos que a parte $P_k$ tem tamanho par, pois $k > 2$. Sendo assim, existe um $i$ tal que $|P_k| = 2i$, onde $0 \leq i \leq d - 1$. Note que $|P_k| \leq 2d - 2$, pois como $P_1$ e $P_2$ possuem tamanho ímpar, temos que há ao menos um termo em $P_1$ e $P_2$. Sabemos que há $\escolhe{2d}{2i}$ maneiras de escolher $2i$ elementos de $2d$ elementos para a parte $P_k$. Note que ao escolher $2i$ elementos para a parte $P_k$ temos que particionar os $2d - 2i$ elementos restantes em $(k-1)$ partes, de modo que apenas as partes $P_1$ e $P_2$ tenham tamanho par. Sendo assim, temos que escolher uma $(k-1)-$partição não par $\mathcal{P'}$ de $2d - 2i$ elementos, onde apenas $P_1$ e $P_2$ tem cardinalidade ímpar. Note que há $\varphi_2(2d - 2i, k-1)$ maneiras de escolher $\mathcal{P'}$. Sendo assim: 
  	  
  	
  	\begin{align}
  		\begin{split}
  			\varphi_2(2d, k) = \sum\limits_{i = 0}^{d - 1}\escolhe{2d}{2i} \cdot \varphi_2(2d - 2i, k-1)
  		\end{split} 
  	\end{align} 
  	
  	Por $HI$, $R(2d - 2i, k-1) = \varphi_2(2d - 2i, k-1)$. Note que $\escolhe{2d}{2i} = \escolhe{2d}{2d - 2i} $. Logo:
  	
  	\begin{align}
  		\begin{split}
  			\varphi_2(2d, k) &= \sum\limits_{i = 0}^{d-1}\escolhe{2d}{2d - 2i} \cdot R(2d - 2i, k-1)\\
  			&= \sum\limits_{i = 1}^{d}\escolhe{2d}{2i} \cdot R(2i, k-1) = R(2d, k)
  		\end{split} 
  	\end{align}  
  	
  	\end{proof} \newl
  
   \begin{lema}  
  	\label{lema4} 
  	$\varphi_2(2n, k) \leq \mu(2n, k)$.
  \end{lema}
  
  \begin{proof} 
  	
  \end{proof} \newpage

  
  \begin{lema}  
  	\label{lema5} 
  	$\Phi_2(2d, k) = \escolhe{k}{2} \cdot \varphi_2(2d, k)$.
  \end{lema}
  
  \begin{proof}
  	Pela definição, $\varphi_2(2d, k)$ conta a quantidade de $k-$partições $\mathcal{P} = \{ P_1, P_2, \ldots P_k \}$, onde apenas $P_1$ e $P_2$ tenham tamanho ímpar. Pela definição, $\Phi_2(2d, k)$ conta a quantidade de $k-$partições $\mathcal{P} = \{ P_1, P_2, \ldots P_k \}$ onde exatamente duas partes quaisquer $P_i$ e $P_j$ tem tamanho ímpar, para algum $1 \leq i, j \leq k$. Note que $\varphi_2(2d, k)$ não conta as $k-$partições onde $|P_i|$ ou $|P_j|$ são ímpares, para $i, j \geq 3$. Mas, neste caso, podemos considerar $P_i$ e $P_j$ como sendo as partes $P_1$ e $P_2$, de modo que a quantidade de $k-$partições ímpares distintas onde somente $|P_i|$ e $|P_j|$ são ímpares seja igual a $\varphi_2(2d, k)$. Como há exatamente $\escolhe{k}{2}$ formas de escolher duas partes $P_i$ e $P_j$ entre as $k$ partes, de modo que $|P_i|$ e $|P_j|$ sejam ímpares, temos que $\Phi_2(2d, k) = \escolhe{k}{2} \cdot \varphi_2(2d, k)$.
  	
  \end{proof}\newl
  
    
  \begin{lema}  
  	\label{lema6} 
  	Seja a recorrência a seguir: \\
  	
  	\begin{equation}
  		X(2d) =
  		\begin{cases}
  			1 & \text{se $d = 0$}\\
  			(2d - 1) k \cdot X(2d - 2) & \text{c. c.} \\ 
  		\end{cases}
  	\end{equation}
  	
  	Temos que $\mu(2d, k) \leq X(2d)$.
  \end{lema}
  
  \begin{proof} 
  	A demonstração segue por indução em $d$. \newl
  	
  	Base $(d = 0)$: Se $d=0$, então $T(0, k) = 1 \leq X(0)$ e o resultado segue. \newl
  	
  	Passo $(d > 0)$: Suponha que $\mu(2\ell, k) \leq X(2\ell)$, para todo $0 \leq \ell < d$. Seja $\mathcal{P}=(P_1, P_2 \ldots P_k)$ uma $k-$partição par de $2d$ elementos. Considere que o elemento $2d \in P_1$. Como $|P_1|$ é par, temos que existe um elemento $y \in P_1$, tal que $y \neq 2d$. Seja $\mathcal{P'}$ a $k-$partição resultante da remoção dos elementos $2d$ e $y$ da parte $P_1$ de $\mathcal{P}$. Note que $\mathcal{P'}$ é uma $k-$partição par de $2d - 2$ elementos. Portanto, $\mathcal{P'}$ é contada em $\mu(2d - 2, k)$. Sendo assim, $\mu(2n - 2, k)$ conta as $k-$partições pares de $2d$ elementos onde $y, 2n \in P_1$. Note que $k \cdot \mu(2n - 2, k)$ conta as $k-$partições pares de $2d$ elementos, onde o elemento $y$ e o elemento $2n$ pertencem a mesma parte, pois há $k$ partes em $\mathcal{P}$ e $\mu(2n - 2, k)$ conta as $k-$partições pares de $2d$ elementos onde $y, 2n \in P_1$. Observe que $y$ pode ser qualquer um dos $2n - 1$ elementos restantes. Sendo assim, $(2d - 1)k \cdot \mu(2d - 2, k)$ conta as $k-$partições pares de $2n$ elementos, onde $y, 2n \in P_i$, onde $P_i \in \mathcal{P}$ e $y$ é um elemento distinto de $2n$, portanto, $\mu(2d, k) \leq (2d - 1)k \cdot \mu(2d - 2, k)$. Por $HI$, $\mu(2d - 2, k) \leq X(2d - 2)$, logo:
  	
  	\begin{align}
  		\begin{split}
  			\mu(2d, k) \leq (2d - 1)k \cdot \mu(2d - 2, k) \leq (2d - 1)k \cdot X(2d - 2) = X(2d)
  		\end{split} 
  	\end{align} 
  	
  \end{proof} \enlargethispage{-3\baselineskip}

  \begin{lema}  
  	\label{lema7} 
  	$ X(2d) \leq (d\cdot k)^d$
  \end{lema}
  
  \begin{proof}
  	
  \end{proof} \newl
  
  \begin{lema} 
  	\label{lema8} 
  	$\Phi(2d+2, k) \geq \Phi(2d, k) \cdot k^{2}$
  \end{lema}
  
  \begin{proof}
  	
  Para demonstrar este lema, iremos contar as $k-$partições não pares possíveis de $2d+2$ elementos geradas a partir das $k-$partições contadas em $\Phi_2(2d, k) $, $\Phi_{\geq 4}(2d, k)$ e em $\mu(2d, k)$. Note que podemos formar uma $k-$partição $\mathcal{P}'$ de $2d+2$ elementos, a partir de uma $k-$partição $\mathcal{P}$ de $2d$ elementos, combinando os elementos $2n+1$ e $2n+2$ entre as $k$ partes de $\mathcal{P}$. Observe que temos $k^2$ maneiras de combinar os elementos $2n+1$ e $2n+2$ entre as $k$ partes, sendo assim, há $k^2$ partições $\mathcal{P}_1'$ distintas.
  	
  Agora, tome uma $k-$partição não par $\mathcal{P}_1$ de $2d$ elementos contada em $\Phi_{\geq 4}(2d, k)$, \ie, $\mathcal{P}_1$ é uma $k-$partição com ao menos $4$ partes de tamanho ímpar. Seja $\mathcal{P}_1'$ uma $k-$partição resultante das $k^2$ combinações dos termos $2d+1$ e $2d+2$ nas $k$ partes de $\mathcal{P}_1$. Como $\mathcal{P}_1$ tem ao menos $4$ partes de tamanho ímpar, temos que $\mathcal{P}_1'$ é uma $k-$partição não par, para qualquer $k-$partição $\mathcal{P}_1'$. Sendo assim, $\Phi(2n+2, k) \geq \Phi_{\geq 4}(2n, k) \cdot k^2$.
  
  	Agora, tome uma $k-$partição não par $\mathcal{P}_2$ de $2d$ elementos, com exatamente duas partes de tamanho ímpar. Considere que as partes $P_i$ e $P_j$ tenham tamanho ímpar em $\mathcal{P}_2$. Seja $\mathcal{P}_2'$ uma $k-$partição de $2d+2$ termos resultante das $k^2$ combinações dos termos $2d+1$ e $2n+2$ nas $k$ partes de $P_2$. Observe que $\mathcal{P}_2'$ é uma $k-$partição par se e somente se combinamos os elementos $2d+1$ e $2d+2$ nas partes $P_i$ e $P_j$. Como há exatamente duas formas de combiná-los de tal maneira, temos que há $k^2 - 2$ partições distintas $\mathcal{P}_2'$ não pares.
  	 Note que a partição $\mathcal{P}_2'$ é distinta da partição $\mathcal{P}_1'$, pois ao retirarmos os elementos $2d+1$ e $2d+2$ de $\mathcal{P}_1'$ e $\mathcal{P}_2'$ obtemos $k-$partições de $2d$ termos distintas. Sendo assim, podemos somar as $k-$partições $\mathcal{P}_2'$ e $\mathcal{P}_1'$. Portanto:
   	 
  \begin{align}
  	\begin{split}
  		\Phi(2d+2, k) &\geq \Phi_{\geq 4}(2d, k) \cdot k^2 + \Phi_{2}(2d, k) \cdot k^2 - 2 \cdot \Phi_{2}(2d, k)\\
  		&\geq (\Phi_{2}(2d, k) + \Phi_{\geq 4}(2d, k)) \cdot k^2 - 2 \cdot \Phi_{2}(2d, k)
  	\end{split} 
  \end{align}  
   
  	 Agora, tome uma $k-$partição par $\mathcal{P}_3$ de $2d$ elementos. Seja $\mathcal{P}_3'$ uma $k-$partição de $2d+2$ termos resultante das $k^2$ combinações possíveis dos elementos $2d+1$ e $2d+2$ nas $k$ partes de $\mathcal{P}_3$. Note que se os elementos $2d+1$ e $2d+2$ pertencem a mesma parte, então $\mathcal{P}_3'$ é uma $k-$partição par. Do contrário, $\mathcal{P}_3'$ é uma $k-$partição ímpar. Sendo assim, temos exatamente $k$ partições $\mathcal{P}_3'$ pares, das $k^2$ partições possíveis, pois há $k$ maneiras dos termos $2n+1$ e $2n+2$ pertencerem a mesma parte de $\mathcal{P}_3$. Logo, temos $k^2 - k$ partições não pares $\mathcal{P}_3'$. Note que $\mathcal{P}_3'$ é distinto de $\mathcal{P}_2'$ e $\mathcal{P}_1'$, pelo mesmo argumento dado anteriormente. Sendo assim: 	
  	
  	 \begin{align}
  		\begin{split}
  			\Phi(2d+2, k) &\geq (\Phi_{2}(2d, k) + \Phi_{\geq 4}(2d, k)) \cdot k^2 - 2 \cdot \Phi_{2}(2d, k) + \mu(2d, k)\cdot k  (k-1)
  		\end{split} 
  	\end{align}  
  	
  	Note que $\Phi(2d, k) = \Phi_2(2d, k) + \Phi_{\geq 4}(2d, k)$, pois, como temos um número par de elementos, não há como ter uma $k-$partição com ímpar partes de tamanho ímpar, sendo assim, $\Phi_i(2d, k) = 0$, para todo $i$ ímpar. Logo: \newl
  	 
  	
  	\begin{align}
  		\begin{split}
  			\Phi(2d+2, k) &\geq (\Phi_{2}(2d, k) + \Phi_{\geq 4}(2d, k)) \cdot k^2 - 2 \cdot \Phi_{2}(2d, k) + \mu(2n, k)\cdot k  (k-1) \\
  			&\geq (\Phi(2d, k)) \cdot k^2 - 2 \cdot \Phi_{2}(2d, k) + \mu(2d, k)\cdot k  (k-1)
  		\end{split} 
  	\end{align}   
  	
  	Iremos demonstrar que $\mu(2d, k)\cdot k  (k-1) - 2 \cdot \Phi_2(2d, k) \geq 0$. Pelo Lema $\ref{lema5}$, temos que: 
  	
  	\begin{align}
  		\begin{split}
  			2 \cdot \Phi_2(2d, k) &= 2 \cdot \escolhe{k}{2} \cdot \varphi_2(2d, k) \\
  			& = k(k-1) \cdot \varphi_2(2d, k)
  		\end{split} 
  	\end{align}  
  	
  	Pelo Lema $\ref{lema4}$, temos que: 
  	
  	\begin{align}
  		\begin{split}
  			k(k-1) \cdot \varphi_2(2d, k) &\leq k(k-1) \cdot \mu(2d, k)
  		\end{split} 
  	\end{align}  
  	
  	Por $(14)$ e $(15)$, temos que:
  	
  	\begin{align}
  		\begin{split}
  			2 \cdot \Phi_2(2d, k) & \leq k(k-1) \cdot \varphi_2(2d, k) \leq k(k-1) \cdot \mu(2d, k) \\
  			& \therefore \mu(2d, k)\cdot k  (k-1) - 2 \cdot \Phi_2(2d, k) \geq 0
  		\end{split} 
  	\end{align} 
  	
  	Sendo assim, temos que $\Phi(2d+2, k) \geq \Phi(2d, k) \cdot k^2$, como desejado.
  	
  \end{proof} \newl
  
  
  \begin{lema}  
  	\label{lema9} 
  	Seja $X_{2d}$ o evento de uma $k-$partição de $2d$ elementos $\mathcal{P}$ ser par. Seja $X_{2d+2}$ o evento de uma $k-$partição de $2d + 2$ elementos $\mathcal{P'}$ ser par. Temos que $\mathds{P}[X_{2d+2}] \leq \mathds{P}[X_{2d}]$.
  \end{lema}
  
  \begin{proof} 
  	
  		Sejam $\Omega$ e $\Omega'$ os conjuntos de todas $k-$partições de $2d$ e $2d+2$ elementos, respectivamente. Note que $\mathds{P}[\overline{X_{2d}}] = \dfrac{\Phi(2d, k)}{|\Omega|}$ e $\mathds{P}[\overline{X_{2d+2}}] = \dfrac{\Phi(2d+2, k)}{|\Omega'|}$. Observe que $|\Omega| = k^{2d}$, pois cada um dos $2d$ termos pode pertencer a qualquer uma das $k$ partes independentemente. Da mesma forma, $|\Omega'| = k^{2d + 2}$. Pelo Lema $\ref{lema8}$:
  	
  	\begin{align}
  		\begin{split}
  		  &	\Phi(2d+2, k) \geq \Phi(2d, k) \cdot k^2 \\
  		  & (\rightarrow) \dfrac{\Phi(2d+2, k)}{k^{2d+2}}	 \geq \dfrac{\Phi(2d, k)}{k^{2d+2}}	 \cdot k^2 \\
  		  & (\rightarrow) \dfrac{\Phi(2d+2, k)}{k^{2d+2}}	 \geq \dfrac{\Phi(2d, k)}{k^{2d}}\\
  		  & (\rightarrow) \mathds{P}[\overline{X_{2d+2}}] \geq \mathds{P}[\overline{X_{2d}}]\\
  		\end{split} 
  	\end{align}
  	
  	 Portanto, temos que $\mathds{P}[{X_{2d+2}}] = 1 - \mathds{P}[\overline{X_{2d+2}}] \leq 1 - \mathds{P}[\overline{X_{2d}}] = \mathds{P}[{X_{2d}}]$.
  	
  \end{proof}\newl 
 
 \begin{lema}  
 	\label{lema10} 
 	Seja $X_{2d}$ o evento de uma $k-$partição de $2d$ elementos $\mathcal{P}$ ser par. Seja $G$ um grafo de ordem $n$ colorido com $k$ cores uniforme e aleatoriamente. Seja $Y_{v}$ o evento de $v$ não ter testemunha ímpar, para $v \in V(G)$ e $d(v) = 2d$. Temos que $\mathds{P}[X_{2d}] = \mathds{P}[Y_v]$.
 \end{lema}
 
 \begin{proof}
 	Sabemos que há $\mu(2d, k)$ maneiras de $k-$colorir os vértices pertencentes a $N(v)$, de modo que $v$ não possua testemunha ímpar. Note que temos exatamente $\mu(2d, k) \cdot k^{n - 2d}$ maneiras de colorir $G$ com $k$ cores de modo que $v$ não possua testemunha ímpar, pois ao colorir $N(v)$ com uma das $k-$colorações contadas em $\mu(2d, k)$, podemos colorir os vértices de $V(G)\setminus N(s)$ com qualquer uma das $k$ cores disponíveis. Note que há $k^n$ formas de colorir $G$ com $k$ cores. Sendo assim: 
 	
 	\begin{align}
 		\begin{split}
 			 \mathds{P}[Y_v] = \dfrac{\mu(2d, k) \cdot k^{n - 2d}}{k^n} = \dfrac{\mu(2d, k)}{k^{2d}} = \mathds{P}[X_{2d}]
 		\end{split} 
 	\end{align}
 	
 \end{proof}
 
 \begin{lema}
 	\label{lema11} Seja $G$ um grafo. Sejam $\delta$ e $\Delta$ o grau mínimo e máximo de $G$, respectivamente. Temos que $\chi_{io}(G) \leq \ell \cdot \sqrt[\ell]{e \cdot \Delta^2}$, para $1 \leq \ell \leq \dfrac{\delta}{2}$.
 \end{lema}
  
  \begin{proof}
  	Pinte os vértices de $G$ com $k$ cores aleatoriamente e independentemente, onde $k = \ell \cdot \sqrt[\ell]{e \cdot \Delta^2}$, para $1 \leq \ell \leq \dfrac{\delta}{2}$. Seja $Y_{v}$ o evento de $v$ não ter testemunha ímpar, para $v \in V(G)$. Se $v$ tem grau ímpar, sabemos que $\mathds{P}[Y_v] = 1$. Suponha que $v$ tem grau par, \ie, $d(v) = 2d$, para $d \geq 0$. Pelo Lema $\ref{lema10}$, temos que $\mathds{P}[Y_v] = \mathds{P}[X_{2d}]$, onde $X_{2d}$ é o evento de uma $k-$partição de $2d$ elementos $\mathcal{P}$ ser par. Pelo Lema $\ref{lema9}$, temos que $\mathds{P}[X_{2d}] \leq \mathds{P}[X_{2\ell}]$, para $\ell \leq d$. Note que $\mathds{P}[X_{2\ell}] =\dfrac{\mu(2\ell, k)}{k^{2\ell}} $. Pelos Lemas $6$ e $7$, temos que $\mu(2\ell, k) \leq (\ell \cdot k)^\ell$. Sendo assim: 
  	
  	\begin{align}
  		\begin{split}
  			\mathds{P}[Y_{v}] \leq \mathds{P}[X_{2\ell}] = \dfrac{\mu(2\ell, k)}{k^{2\ell}} \leq  \dfrac{(l \cdot k)^\ell}{k^{2\ell}} = \left(\dfrac{\ell}{k}\right)^\ell = \dfrac{1}{e \cdot \Delta^2}
  		\end{split} 
  	\end{align}
  	
  	Note que o evento $Y_v$ é dependente a no máximo $\delta^2$ eventos, para todo $v \in V(G)$. Sendo assim, pelo Lema Local de Lovász, temos que $\mathds{P}[\bigcap\limits_{v \in V(G)} \overline{Y_{v}} ] > 0$. Logo $\chi_{io} \leq k = \ell \cdot \sqrt[\ell]{e \cdot \Delta^2}$, para $1 \leq \ell \leq \dfrac{\delta}{2}$.
  	
  \end{proof}
\end{document}
 