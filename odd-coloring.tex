\documentclass[12pt]{article}
\usepackage{cancel}
\usepackage{sbc-template/sbc-template}
\usepackage{graphicx,url}
\usepackage[utf8]{inputenc}
\usepackage[brazil]{babel}    
\usepackage{geometry}
\usepackage{setspace} 
\usepackage[T1]{fontenc}
\usepackage[brazil]{babel}
\usepackage{amsthm}
\usepackage[portuguese, ruled, lined, linesnumbered, commentsnumbered, longend]{algorithm2e}
\usepackage{blindtext}
\usepackage{geometry}
\usepackage{scrextend}
\usepackage{setspace}
\usepackage{amssymb}
\usepackage{amsmath}
\usepackage{dsfont}
 
\newcommand{\mycomment}[1]{}
\newenvironment{citacao}{%
\begin{list}{}{%
\setlength{\topsep}{1pt}%
\setlength{\leftmargin}{4cm}%
\setlength{\rightmargin}{0cm}%
\setlength{\listparindent}{\parindent}%
\setlength{\itemindent}{\parindent}%
\setlength{\parsep}{\parskip}%
}%
\item[]}{\end{list}}

%\renewcommand{\baselinestretch}{0.6} 

\sloppy
\raggedbottom
 
\begin{document} 
 \def\changemargin#1#2{\list{}{\rightmargin#2\leftmargin#1}\item[]} 
\let\endchangemargin=\endlist 
\newcommand{\wbigcup}{\mathop{\widetilde{\bigcup}}\displaylimits}

\newcommand{\ie}{\textit{i}.\textit{e}.} 
\newcommand{\blceil}{\left \lceil}
\newcommand{\brceil}{\right \rceil} 
\newcommand{\blfloor}{\left \lfloor}
\newcommand{\brfloor}{\right \rfloor} 
\newcommand{\defcomunidade}{Def.1 \ (a) \text{,} \ (b) \ \text{e} \ (c)}

\newcommand{\newbegin}{\vspace{0.5cm}}

\newcommand{\newl}{\vspace{0.1cm}}
\newcommand{\escolhe}[2]{\resizebox{!}{15pt}{$\displaystyle \binom{#1}{#2}$}}

\newtheorem{definicao}{Def} 
\newtheorem{exemplo}{Exemplo}  
\newtheorem{lema}{Lema} 
\newtheorem{teo}{Teorema} 
\newtheorem{cor}{Corolário}
\newtheorem{prop}{Proposição}
\newtheorem{defi}{Definição}

%LEMA 
%LEMA 
%LEMA 
%LEMA


%LEMA 1 
%LEMA 1 
%LEMA 1 
%LEMA 1  
\begin{prop}
	\label{prop1}
	Seja $G$ um grafo conexo. Seja $T$ uma árvore de Busca em Largura de $G$ a partir de um vértice $v$ qualquer. Se existem vértices $s, t \in V(G)$ tais que $st \in E(G)$, $st \notin E(T)$ e $dist_T(v, s) = dist_T(v, t) + 1$, então $G$ possui ciclo par.
\end{prop} \newbegin

\begin{prop}
	\label{prop2}
	Seja $G$ um grafo conexo livre de ciclos pares. Seja $T$ uma Árvore de Busca em Largura de $G$ a partir de $v \in V(G)$. Seja $V \subsetneq V(G)$ o conjunto de vértices com distância $p > 0$ de $v$. Temos que o conjunto $E(G[V])$ é um emparelhamento.
\end{prop} \newbegin


\begin{lema}  
\label{lema1}
Seja $G$ um grafo. Se existe uma $k$-partição $V_1, V_2, \ldots ,V_k$ dos vértices de $G$ tal que, para todo vértice $v \in V(G)$, temos que $|N(v) \cap V_i| = 1$, para algum $1 \leq i \leq k$, então $\chi_{pcf}(G) \leq  \sum\limits_{i = 1}^{k} \chi(G[V_i])$.
\end{lema}

\begin{proof}
	
 Seja $H_i = G[V_i]$, para todo $1 \leq i \leq k$. Iremos colorir cada subgrafo $H_i$ com $\chi(H_i)$ cores distintas. Para isso, cada cor será representada por um par ordenado. Seja $c_i: V(H_i) \rightarrow \{ i \} \times \chi(H_i)$ uma coloração própria de $H_i$. Para todo par distinto de colorações $c_i$ e $c_j$, temos que $c_i(v) \neq c_j(u)$, para todo $ v \in V(H_i)$ e $ u \in V(H_j)$, pois $(i, x) \neq (j, y)$ para $i \neq j$.\newl
  
  Seja $c$ uma coloração de $G$ tal que $c(v) = c_i(v)$ se e somente se $v \in V(H_i)$. Em outras palavras, $c$ é a união das colorações usadas em cada subgrafo $H_i$. Como $c_i$ é uma coloração própria de $H_i$ e todo par distinto de subgrafos $H_i$ e $H_j$ são coloridos com cores distintas, temos que $c$ é uma coloração própria de $G$. \newl
 
  Como, para todo vértice $v \in V(G)$, vale que $|N(v) \cap V(H_i)| = 1$, para algum subgrafo $H_i$, e como as cores usadas em $H_i$ são distintas das cores usadas em $V(G)\setminus V(H_i)$, temos que existe uma cor $(i, x)$ que aparece uma única vez na vizinhança de $v$, para $x \in [\chi(H_i)]$. Sendo assim, $c$ descreve uma coloração própria livre de conflitos de $G$. Como $c_i$ utiliza $\chi(H_i)$ cores, para todo $1 \leq i \leq k$, temos que $\chi_{pcf}(G) \leq \sum\limits_{i = 1}^{k} \chi(H_i)$. 
  
\end{proof} 


 \begin{teo}
 \label{teo1}
 	Seja $G$ um grafo conexo. Se $G$ é livre de ciclos pares, então $\chi_{pcf}(G) \leq 7$.
 \end{teo}
 
 \begin{proof}
 	Seja $T$ uma Árvore de Busca em Largura de $G$ a partir de um vértice $r$ qualquer. Sabemos que $T$ é uma árvore geradora, pois $G$ é conexo. Seja $V_0, V_1$, $V_2$ uma partição de $G$ tal que $x \in V_i$ se e somente se $i = dist_T(r, x) \pmod{3} $. Seja $s$ um vértice de $G$, tal que $s \neq r$. Seja $p$ o pai de $s$ em $T$ e seja $f$ um filho de $s$ em $T$.
 	Note que $p$ e $f$ pertencem a partições distintas, pois:
 	\begin{align}
 		\begin{split}
 			dist_T(r, p) \pmod{3} \neq dist_T(r, p) + 2 \pmod{3} = dist_T(r, f) \pmod{3}
 		\end{split} 
 	\end{align} 
   
 	 Seja $t \in V(G)$ um vértice tal que $dist_T(r, s) > dist_T(r, t) + 1$. Sabemos que $st \notin E(G)$, pois $T$ é uma árvore de Busca em Largura. Sendo assim, se $st \in E(G)$ e $st \notin E(T)$, então $dist_T(r, s) = dist_T(r, t) + 1$ ou $dist_T(r, s) = dist_T(r, t)$. Pela Proposição $\ref{prop1}$, sabemos que se $st \in E(G)$ e $st \notin E(T)$, então $dist_T(r, s) = dist_T(r, t)$, pois $G$ é livre de ciclos pares. Note que isto implica que $s$ é adjacente a precisamente um vértice $u$ em $G$ tal que $dist_T(r, s) = dist_T(r, u) + 1$, e, sendo assim, $u$ é o pai de $s$ em $T$, \ie, $u = p$. Note que $|N(s) \cap V_i| = 1$, onde $f \in V_i$, para todo $s \in V(G) \setminus\{r\}$. \newl
 	 
 	 Resta agora a partição do vértice raiz $r$. Iremos remover um vértice $v \in N(r)$ da partição $V_1$ e iremos construir uma nova partição $V_0, V^{'}_1, V_2, V_3$ de $G$, de modo que $V^{'}_1 = V_1\setminus\{v\}$ e $V_3 = \{v\}$. Seja $s$ um vértice onde $|N(s) \cap V_1| = 1$, \ie, o pai $p$ de $s$ pertence a $V_1$. Queremos argumentar que a propriedade é satisfeita para $s$ na nova partição $V_0, V^{'}_1, V_2, V_3$. Se $p \neq v$, então $|N(s) \cap V_1'| = 1$ e a propriedade continua valendo. Se $p = v$, então $N(s) \cap V_3 = \{v\}$, \ie, $ |N(s) \cap V_3| = 1$ e a propriedade vale. \newl
 	 
 	 Note que a partição $V_0, V_1', V_2 \text{ e } V_3$ satisfaz a condição do Lema $\ref{lema1}$. Note que pela Proposição $\ref{prop2}$, $E(G[V_i])$ é um emparelhamento. Sendo assim, temos que $\chi(G[V_i]) = 2$, para $0 \leq i \leq 2$. Note que $\chi(G[V_3]) = 1$, pois $V_3 = \{v\}$. Sendo assim, pelo Lema $\ref{lema1}$, temos que $\chi_{pcf}(G) \leq 7$.
  
 \end{proof}  \newpage
 
 \begin{defi}
 	\label{defi1}
 	Seja $P = \{C_1, C_2, \ldots C_k \}$ uma k-partição de $n$ termos. Dizemos que $P$ é uma k-partição par se, para todo $1 \leq i \leq k$, $|C_i|$ é par. Do contrário, dizemos que $P$ é uma k-partição ímpar, \ie, $P$ é uma k-partição ímpar se alguma parte $C_i$ tem tamanho ímpar. 
 \end{defi} \newbegin
 
 \begin{defi}
 	\label{defi2}
 	Seja $P = \{C_1, C_2, \ldots C_k \}$ uma k-partição ímpar de $n$ termos. Dizemos que $P$ é uma k-partição $\ell$-ímpar se há exatamente $\ell$ partes distintas em $P$ de tamanho ímpar, para $\ell \geq 1$.
 \end{defi} \newbegin
 
 \begin{defi}
 	\label{defi3}
 	Denotamos por $\mu(n, k)$ a quantidade de k-partições pares distintas de $n$ termos.
 \end{defi} \newbegin
 
  \begin{defi}
 	\label{defi4}
 	Denotamos por $\Phi(n, k)$ a quantidade de k-partições ímpares distintas de $n$ termos. Denotamos por $\Phi_\ell(n, k)$ a quantidade de k-partições $\ell$-ímpares distintas de $n$ termos, para $\ell \geq 1$. Também denotamos por $\Phi_{\geq \ell}(n, k)$ a quantidade de k-partições $t$-ímpares distintas $n$ termos, para todo $ \ell \leq t \leq k$, \ie, $\Phi_{\geq \ell}(n, k) = \sum\limits_{i = \ell}^{k} \Phi_{i}(n, k)$. 
 \end{defi} \newbegin
 
   \begin{defi}
 	\label{defi5}
   Denotamos por $\varphi_\ell(n, k)$ a quantidade de k-partições $\ell$-ímpares, com a restrição de que somente as primeiras $\ell$ partes tenham cardinalidade ímpar. Claramente temos que $\varphi_\ell(n, k) \leq \Phi_\ell(n, k)$, pois $\varphi_\ell(n, k)$ conta apenas as k-partições com as primeiras $\ell$ partes de cardinalidade ímpar, já $\Phi_\ell(n, k)$ conta qualquer subconjunto pertencente a $\escolhe{[k]}{\ell}$ com cardinalidade ímpar.
 \end{defi} \newbegin
 
 
 \begin{lema}  
 	\label{lema2}
 	Seja a recorrência a seguir: \\
 	
 		\begin{equation}
 		T(2n, k) =
 		\begin{cases}
 			1 & \text{se $n = 0$ ou $k = 1$} \\
 			\sum\limits_{i = 0}^{n} \escolhe{2n}{2i} \cdot T(2i, k-1) & \text{c.c.} \\ 
 		\end{cases}
 	\end{equation}
 	
 	Temos que $T(2n, k) = \mu(2n, k)$.
 \end{lema}
 
 \begin{proof}
 	A demonstração segue por indução em $k$. \newl
 	
 	Base ($k = 1$): Para $k = 1$, temos que $T(2n, k) = \mu(2n, 1) = 1$, pois há uma única partição $P=\{C_1\}$ de $2n$, de modo que $|C_1|$ seja par. Sendo assim, o resultado segue. \newl
 	
 	Passo ($k > 1$): Suponha que $T(2n, \ell) = \mu(2n, \ell)$, para todo $1 \leq \ell < k$. Seja $a_{2i}$ a quantidade de maneiras de escolher $2i$ termos de $2_n$ termos para a parte $C_k$. Como $C_k$ tem tamanho par, temos que $|C_k| = 2i$, para $0 \leq i \leq n$. Note que ao escolher $2i$ termos para a parte $C_k$ temos que escolher uma (k-1)-partição par $P'$ para os $2n - 2i$. Por definição, a quantidade de (k-1)-partições pares distintas de $2n - 2i$ termos é igual a $\mu(2n - 2i, k-1)$. Sendo assim: 
 	
 	\begin{align}
 		\begin{split}
 			\mu(2n, k) = \sum\limits_{i = 0}^{n}a_{2i} \cdot \mu(2n - 2i, k-1)
 		\end{split} 
 	\end{align} 
 	
 	Por $HI$, $T(2n - 2i, k-1) = \mu(2n - 2i, k-1)$. Note que $a_{2i} = \escolhe{2n}{2i} = \escolhe{2n}{2n - 2i} $. Logo:
 	
 	\begin{align}
 		\begin{split}
 			\mu(2n, k) &= \sum\limits_{i = 0}^{n}\escolhe{2n}{2n - 2i} \cdot T(2n - 2i, k-1)\\
 			&= \sum\limits_{i = 0}^{n}\escolhe{2n}{2i} \cdot T(2i, k-1) = T(2n, k)
 		\end{split} 
 	\end{align}
 	
 \end{proof} \newl
 
 
  \begin{lema}  
 	\label{lema3}
 	Seja a recorrência a seguir: \\
 	
 	\begin{equation}
 		R(2n, k) =
 		\begin{cases}
 			0 & \text{se $k \leq 1$}\\
 			2^{2n - 1} & \text{se $k = 2$} \\
 			\sum\limits_{i = 1}^{n} \escolhe{2n}{2i} \cdot R(2i, k-1) & \text{se $k > 2$} \\ 
 		\end{cases}
 	\end{equation}
 	
 	Temos que $R(2n, k) = \varphi_2(2n, k)$.
 \end{lema}
  
  \begin{proof}
  	A demonstração segue por indução em $k$.
  	
  	Base ($k = 2$): Note que $\varphi_2(2n, 2) = \Phi(2n, 2)$, pois, seja uma k-partição ímpar $P=(C_1, C_2)$, como $2n$ é par, temos que $|C_1| \text{ e } |C_2|$ são pares ou $|C_1| \text{ e } |C_2|$ são ímpares. Note que $\Phi(2n, 2) = 2^{2n} - \mu(2n, 2)$, pois há exatamente $2^{2n}$ formas de particionar $2n$ termos em duas partes $C_1$ e $C_2$ e dessas $2^{2n}$ maneiras há $\mu(2n, 2)$ maneiras de particionar $2n$ termos tal que $|C_1|$ e $|C_2|$ sejam pares. Note que:  
  	
  	\begin{align}
  		\begin{split}
  			\mu(2n, 2) &= \sum\limits_{i = 0}^{n}\escolhe{2n}{2i} \cdot T(2i, 1) = \sum\limits_{i = 0}^{n}\escolhe{2n}{2i} = 2^{2n-1}
  		\end{split} 
  	\end{align} 
  	
  	Logo, temos que $\varphi_2(2n, 2) = \Phi(2n, 2) = 2^{2n} - 2^{2n - 1} = 2^{2n - 1} = R(2n, 2)$ e o resultado segue. \newl
  	
  	
  	Passo ($k > 1$): Suponha que $R(2n, \ell) = \varphi_2(2n, \ell)$, para todo $2 \leq \ell < k$. Seja $a_{2i}$ a quantidade de maneiras de escolher $2i$ termos de $2_n$ termos para a parte $C_k$. Como somente as partes $C_1$ e $C_2$ tem tamanho ímpar, temos que a parte $C_k$ tem tamanho par, pois $k > 2$. Logo $|C_k| = 2i$, para $0 \leq i \leq n - 1$. Note que $|C_k| \leq 2n - 2$, pois como $C_1$ e $C_2$ possuem tamanho ímpar, temos que há ao menos um termo em $C_1$ e $C_2$. Note que ao escolher $2i$ termos para a parte $C_k$ temos que escolher uma (k-1)-partição 2-ímpar $P'$ para os $2n - 2i$. Por definição, a quantidade de (k-1)-partições 2-ímpares distintas de $2n - 2i$ termos é igual a $\varphi_2(2n - 2i, k-1)$. Sendo assim: 
  	
  	\begin{align}
  		\begin{split}
  			\varphi_2(2n, k) = \sum\limits_{i = 0}^{n - 1}a_{2i} \cdot \varphi_2(2n - 2i, k-1)
  		\end{split} 
  	\end{align} 
  	
  	Por $HI$, $R(2n - 2i, k-1) = \varphi_2(2n - 2i, k-1)$. Note que $a_{2i} = \escolhe{2n}{2i} = \escolhe{2n}{2n - 2i} $. Logo:
  	
  	\begin{align}
  		\begin{split}
  			\varphi_2(2n, k) &= \sum\limits_{i = 0}^{n-1}\escolhe{2n}{2n - 2i} \cdot R(2n - 2i, k-1)\\
  			&= \sum\limits_{i = 1}^{n}\escolhe{2n}{2i} \cdot R(2i, k-1) = R(2n, k)
  		\end{split} 
  	\end{align}  
  	
  	\end{proof} \newl
  
   \begin{lema}  
  	\label{lema4} 
  	$\varphi_2(2n, k) \leq \mu(2n, k)$.
  \end{lema}
  
  \begin{proof}
  	A demonstração segue por indução em $k$.
  	
  	Base ($k = 2$): Pelo Caso Base do Lema $\ref{lema3}$, temos que $\varphi_2(2n, 2) = 2^{2n - 1} = \mu(2n, 2)$ e o resultado segue. \newl
  	
  	Passo ($k > 2$): Suponha que $\varphi_2(2n, \ell) \leq \mu(2n, \ell)$, para $2 \leq \ell < k$. Note que: \newl
  	
  	\begin{align}
  		\begin{split}
  			\mu(2n, k) = \sum\limits_{i = 0}^{n} \escolhe{2n}{2i} \cdot \mu(2i, k-1)  \geq \sum\limits_{i = 1}^{n} \escolhe{2n}{2i} \cdot \mu(2i, k-1)
  		\end{split} 
  	\end{align}
  	
  	Por $HI$: \newl
  	
  	\begin{align}
  		\begin{split}
  			 \sum\limits_{i = 1}^{n} \escolhe{2n}{2i} \cdot \varphi_2(2i, k-1) \leq \sum\limits_{i = 1}^{n} \escolhe{2n}{2i} \cdot \mu(2i, k-1)
  		\end{split} 
  	\end{align}
  	
  	Como $\varphi_2(2n, k) = \sum\limits_{i = 1}^{n} \escolhe{2n}{2i} \cdot \varphi_2(2i, k-1)$, por $(9)$ e $(10)$, temos que $\varphi_2(2n, k) \leq \mu(2n, k)$.
  
  \end{proof} \newpage

  
  \begin{lema}  
  	\label{lema5} 
  	$\Phi_2(2n, k) = \escolhe{k}{2} \cdot \varphi_2(2n, k)$.
  \end{lema}
  
  \begin{proof}
  	Pela definição, $\varphi_2(2n, k)$ conta a quantidade de k-partições $C_1, C_2 \ldots C_k$ onde apenas $C_1$ e $C_2$ tenham tamanho ímpar. Pela definição, $\Phi_2(2n, k)$ conta a quantidade de k-partições $C_1, C_2 \ldots C_k$ onde exatamente duas partes quaisquer $C_i$ e $C_j$ tem tamanho ímpar, para $1 \leq i, j \leq k$. Note que $\varphi_2(2n, k)$ não conta as k-partições onde $|C_i|$ ou $|C_j|$ são ímpares, para $i, j \geq 3$. Mas, neste caso, podemos considerar $C_i$ e $C_j$ como sendo as partes $C_1$ e $C_2$, de modo que a quantidade de k-partições ímpares distintas onde somente $|C_i|$ e $|C_j|$ são ímpares seja igual a $\varphi_2(2n, k)$. Como há exatamente $\escolhe{k}{2}$ formas de escolher duas partes $C_i$ e $C_j$ entre as $k$ partes, de modo que $|C_i|$ e $|C_j|$ sejam ímpares, temos que $\Phi_2(2n, k) = \escolhe{k}{2} \cdot \varphi_2(2n, k)$.
  	
  \end{proof}\newl
  
    
  \begin{lema}  
  	\label{lema9} 
  	Seja a recorrência a seguir: \\
  	
  	\begin{equation}
  		X(2n) =
  		\begin{cases}
  			1 & \text{se $n = 0$}\\
  			(2n - 1) k \cdot X(2n - 2) & \text{c. c.} \\ 
  		\end{cases}
  	\end{equation}
  	
  	Temos que $\mu(2n, k) \leq X(2n)$.
  \end{lema}
  
  \begin{proof} 
  	A demonstração segue por indução em $n$. \newl
  	
  	Base $(n = 0)$: Se $n=0$, então $T(0, k) = 1 \leq X(0)$ e o resultado segue. \newl
  	
  	Passo $(n > 0)$: Suponha que $\mu(2\ell, k) \leq X(2\ell)$, para todo $0 \leq \ell < n$. Seja $P=(C_1, C_2 \ldots C_k)$ uma k-partição par de $2n$ termos. Considere que o termo $2n \in C_i$, para algum $1 \leq i \leq k$. Como $|C_i|$ é par, temos que existe um termo $y \in C_i$, tal que $y \neq 2n$. Seja $P'$ a k-partição resultante da remoção dos termos $2n$ e $y$ da parte $C_i$ de $P$. Note que $P'$ é uma k-partição par de $2n - 2$ termos. Portanto, $P'$ é contada em $\mu(2n - 2, k)$. Sendo assim, $\mu(2n - 2, k)$ conta as k-partições pares de $2n$ termos onde $y, 2n \in C_i$, para algum $1 \leq i \leq k$. Como há $k$ possibilidades para a parte $C_i$, temos que $k \cdot \mu(2n - 2, k)$ conta as k-partições pares de $2n$ termos onde $y, 2n \in C_j$, para todo $1 \leq j \leq k$. Observe que $y$ pode ser qualquer um dos $2n - 1$ termos restantes. Sendo assim, cada k-partição par $P$ de $2n$ termos é equivalente a alguma combinação de $(2n - 1)k \cdot \mu(2n - 2, k)$. Sendo assim, $\mu(2n, k) \leq (2n - 1)k \cdot \mu(2n - 2, k)$. Por $HI$, $\mu(2n - 2, k) \leq X(2n - 2)$, logo:
  	
  	\begin{align}
  		\begin{split}
  			\mu(2n, k) \leq (2n - 1)k \cdot \mu(2n - 2, k) \leq (2n - 1)k \cdot X(2n - 2) = X(2n)
  		\end{split} 
  	\end{align} 
  	
  \end{proof}\newl
  
   \begin{lema}  
  	\label{lema6} 
  	Seja $X_\ell$ o evento de uma k-partição de $\ell$ termos $P$ ser ímpar. Temos que $\mathds{P}[X_{2n+2}] \geq \mathds{P}[X_{2n}] \longleftrightarrow \Phi(2n+2, k) \geq \Phi(2n, k) \cdot k^{2}$.
  \end{lema}
  
  \begin{proof}
  	Sejam $\Omega$ e $\Omega'$ os conjuntos de todas k-partições de $2n$ e $2n+2$ termos, respectivamente. Temos que $\mathds{P}[X_{2n}] = \dfrac{\Phi(2n, k)}{|\Omega|}$ e $\mathds{P}[X_{2n+2}] = \dfrac{\Phi(2n+2, k)}{|\Omega'|}$. Note que $|\Omega| = k^{2n}$, pois cada um dos $2n$ termos pode pertencer a qualquer um dos $k$ conjuntos independentemente. Da mesma forma, $|\Omega'| = k^{2n + 2}$. Sendo assim:
  	
  	\begin{align}
  		\begin{split}
  			\mathds{P}[X_{2n+2}] \geq {P}[X_{2n}] &\longleftrightarrow  \dfrac{\Phi(2n+2, k)}{k^{2n + 2}} \geq \dfrac{\Phi(2n, k)}{k^{2n}}\\
  			&\longleftrightarrow \Phi(2n+2, k) \geq \Phi(2n, k) \cdot k^{2}
  		\end{split} 
  	\end{align}
  	
  \end{proof}\newl
 
  
  \begin{lema}
  	\label{lema7} 
  	$\Phi(2n+2, k) \geq \Phi(2n, k) \cdot k^{2}$
  \end{lema}
  
  \begin{proof}
  	
  	Note que $\Phi(2n, k) = \Phi_2(2n, k) + \Phi_{\geq 4}(2n, k)$, pois, como temos um número par de termos, não há como ter uma k-partição com ímpar partes de tamanho ímpar, sendo assim, $\Phi_i(2n, k) = 0$, para todo $i$ ímpar. Para demonstrar este lema, iremos contar as k-partições ímpares possíveis de $2n+2$ termos geradas a partir das k-partições contadas em $\Phi_2(2n, k) $, $\Phi_{\geq 4}(2n, k)$ e em $\mu(2n, k)$. \newl
  	
  	Tome uma k-partição $\ell$-ímpar $P_1$ de $2n$ termos, onde $\ell \geq 4$, \ie, $P_1$ é uma k-partição contada em $ \Phi_{\geq 4}(2n, k)$. Note que o termo $2n+1$ possui $k$ possibilidades de partes para formar uma nova k-partição a partir de $P_1$, pois o termo $2n+1$ pode pertencer a qualquer uma das $k$ partes de $P_1$. O mesmo vale para o termo $2n+2$. Sendo assim, temos $k^2$ k-partições $P_1'$ distintas de $2n+2$ termos possíveis a partir da partição $P_1$. Note que toda nova partição $P_1'$ resultante é uma k-partição ímpar, pois temos ao menos $4$ partes ímpares em $P_1$. Como temos $\Phi_{\geq 4}(2n, k)$ k-partições ímpares distintas com ao menos $4$ partes ímpares, temos que $\Phi(2n+2, k) \geq \Phi_{\geq 4}(2n, k) \cdot k^2$. \newl
  	
  	Agora, tome uma k-partição $2$-ímpar $P_2$ de $2n$ termos. Considere que as partes $C_i$ e $C_j$ tenham tamanho ímpar em $P_2$. Seja $P_2'$ uma k-partição de $2n+2$ termos resultante das $k^2$ combinações dos termos $2n+1$ e $2n+2$ nas $k$ partes de $P_2$. Se o termo $2n+1$ ou o termo $2n+2$ pertencer a alguma parte $C_x$ em $P_2$, onde $x \neq i, j$, então $P_2'$ é uma k-partição ímpar, pois $|C_x|$ é par, logo $|C_x \cup \{2n+1\}|$ é ímpar. Se ambos termos $2n+1$ e $2n+2$ pertencem à parte $C_i$, então $P_2'$ é uma k-partição ímpar, pois $|C_i \cup \{2n+1, 2n+2\}|$ é ímpar. O mesmo vale para a parte $C_j$. Se os termos $2n+1$ e $2n+2$ pertencem às partes $C_i$ e $C_j$, respectivamente ou não, então $P_2'$ é uma k-partição par, pois $C_i$ e $C_j$ são as únicas partes de tamanho ímpar em $P_2$. Sendo assim, temos $k^2 - 2$ k-partições ímpares $P_2'$ distintas a partir de $P_2$. Logo $\Phi(2n+2, k) \geq \Phi_{2}(2n, k) \cdot k^2 - 2 \cdot \Phi_{2}(2n, k)$. Note que a partição $P_2'$ resultante de $P_2$ é distinta da partição $P_1'$ resultante de $P_1$, pois ao retirarmos os termos $2n+1$ e $2n+2$ de $P_1'$ e $P_2'$ obtemos $P_1$ e $P_2$, e sabemos que $P_1$ e $P_2$ são k-partições distintas, pois $P_1$ tem exatamente duas partes de tamanho ímpar e $P_2$ tem pelo menos quatro partes de tamanho ímpar. Sendo assim, podemos somar as partições $P_2'$ obtidas de $P_2$ junto com as partições $P_1'$ obtidas de $P_1$. Portanto:
  	
  	\begin{align}
  		\begin{split}
  			\Phi(2n+2, k) &\geq \Phi_{\geq 4}(2n, k) \cdot k^2 + \Phi_{2}(2n, k) \cdot k^2 - 2 \cdot \Phi_{2}(2n, k) \\
  			&\geq \Phi(2n, k) \cdot k^2 - 2 \cdot \Phi_2(2n, k)
  		\end{split} 
  	\end{align}  

  	Agora, tome uma k-partição par $P_3$ de $2n$ termos. Seja $P_3'$ uma k-partição de $2n+2$ termos resultante das $k^2$ combinações possíveis dos termos $2n+1$ e $2n+2$ nas $k$ partes de $P_3$. Note que se os termos $2n+1$ e $2n+2$ pertencem a mesma parte, então $P_3'$ é uma k-partição par. Do contrário, $P_3'$ é uma k-partição ímpar. Sendo assim das $k^2$ combinações possíveis, temos que exatamente $k$ combinações são k-partições pares, pois há $k$ maneiras dos termos $2n+1$ e $2n+2$ pertencerem a mesma parte de $P_3$. Logo, temos $k^2 - k$ partições ímpares $P_3'$ resultantes de $P_3$. Note que $P_3'$ é distinto de $P_2'$ e $P_1'$, pelo mesmo argumento dado anteriormente. Sendo assim: 	
  	
  	\begin{align}
  		\begin{split}
  			\Phi(2n+2, k) &\geq \Phi(2n, k) \cdot k^2 + \mu(2n, k)\cdot k^2 - \mu(2n, k) \cdot k - 2 \cdot \Phi_2(2n, k) \\
  			&\geq  \Phi(2n, k) \cdot k^2 + \mu(2n, k)\cdot k  (k-1) - 2 \cdot \Phi_2(2n, k)
  		\end{split} 
  	\end{align}  
  	 
  	 Iremos demonstrar que $\mu(2n, k)\cdot k  (k-1) - 2 \cdot \Phi_2(2n, k) \geq 0$. Pelo Lema $\ref{lema5}$, temos que: 
  	 
  	 \begin{align}
  	 	\begin{split}
  	 	 2 \cdot \Phi_2(2n, k) &\leq 2 \cdot \escolhe{k}{2} \cdot \varphi_2(2n, k) \\
  	 	 & \leq k(k-1) \cdot \varphi_2(2n, k)
  	 	\end{split} 
  	 \end{align}  
  	 
  	 Pelo Lema $\ref{lema4}$, temos que: 
  	 
  	 \begin{align}
  	 	\begin{split}
  	 		 k(k-1) \cdot \varphi_2(2n, k) &\leq k(k-1) \cdot \mu(2n, k)
  	 	\end{split} 
  	 \end{align}  
  	 
  	 Por $(14)$ e $(15)$, temos que:
  	 
  	 \begin{align}
  	 	\begin{split}
  	 		2 \cdot \Phi_2(2n, k) & \leq k(k-1) \cdot \varphi_2(2n, k) \leq k(k-1) \cdot \mu(2n, k) \\
  	 		& \therefore \mu(2n, k)\cdot k  (k-1) - 2 \cdot \Phi_2(2n, k) \geq 0
  	 	\end{split} 
  	 \end{align} 
  	 
  	 Sendo assim, temos que $\Phi(2n+2, k) \geq \Phi(2n, k) \cdot k^2$, como desejado.
  	 
  \end{proof} \newl
  
   \begin{lema}  
  	\label{lema8} 
  	Seja $Y_\ell$ o evento de uma k-partição de $\ell$ termos $P$ ser par. Temos que $\mathds{P}[Y_{2n+2}] \leq \mathds{P}[Y_{2n}]$.
  \end{lema}
  
  \begin{proof} 
  	Por definição, $P$ é uma k-partição par se $|C_i|$ é par, para toda parte $C_i$. Em contrapartida, $P$ é uma k-partição ímpar se $|C_j|$ é ímpar, para alguma parte $C_j$ de $P$. Sendo assim, $\overline{X_\ell} = Y_\ell$, onde $X_\ell$ o evento de uma k-partição de $\ell$ termos $P$ ser ímpar. Logo, $\mathds{P}[Y_{\ell}] = 1 - \mathds{P}[X_{\ell}]$. Portanto: 
  	
  	 \begin{align}
  		\begin{split}
  			 \mathds{P}[Y_{2n}] &= 1 - \mathds{P}[X_{2n}] \ \    \\ 
  			 \mathds{P}[Y_{2n+2}] &= 1 - \mathds{P}[X_{2n+2}]
  		\end{split} 
  	\end{align} 
  	
  	Pelos Lemas $\ref{lema6}$ e $\ref{lema7}$, temos que $\mathds{P}[X_{2n+2}] \geq \mathds{P}[X_{2n}]$, logo: 
  	
  	\begin{align}
  		\begin{split} 
  			\mathds{P}[Y_{2n+2}] &= 1 - \mathds{P}[X_{2n+2}] \leq 1 -  \mathds{P}[X_{2n}] = \mathds{P}[Y_{2n}]
  		\end{split} 
  	\end{align} 
  	
  \end{proof}\newl
  
\end{document}
 