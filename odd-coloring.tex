\documentclass[12pt]{article}
\usepackage{cancel}
\usepackage{sbc-template/sbc-template}
\usepackage{graphicx,url}
\usepackage[utf8]{inputenc}
\usepackage[brazil]{babel}    
\usepackage{geometry}
\usepackage{setspace} 
\usepackage[T1]{fontenc}
\usepackage[brazil]{babel}
\usepackage{amsthm}
\usepackage[portuguese, ruled, lined, linesnumbered, commentsnumbered, longend]{algorithm2e}
\usepackage{blindtext}
\usepackage{geometry}
\usepackage{scrextend}
\usepackage{setspace}
\usepackage{amssymb}
\usepackage{amsmath}
\usepackage{dsfont}
\usepackage{comment}
 
\newcommand{\mycomment}[1]{}
\newenvironment{citacao}{%
\begin{list}{}{%
\setlength{\topsep}{1pt}%
\setlength{\leftmargin}{4cm}%
\setlength{\rightmargin}{0cm}%
\setlength{\listparindent}{\parindent}%
\setlength{\itemindent}{\parindent}%
\setlength{\parsep}{\parskip}%
}%
\item[]}{\end{list}}

%\renewcommand{\baselinestretch}{0.6} 

\sloppy
\raggedbottom
 
\begin{document} 
 \def\changemargin#1#2{\list{}{\rightmargin#2\leftmargin#1}\item[]} 
\let\endchangemargin=\endlist 
\newcommand{\wbigcup}{\mathop{\widetilde{\bigcup}}\displaylimits}

\newcommand{\overbar}[1]{\mkern 1.5mu\overline{\mkern-1.5mu#1\mkern-1.5mu}\mkern 1.5mu}
\newcommand{\ie}{\textit{i}.\textit{e}.} 
\newcommand{\blceil}{\left \lceil}
\newcommand{\brceil}{\right \rceil} 
\newcommand{\blfloor}{\left \lfloor}
\newcommand{\brfloor}{\right \rfloor} 
\newcommand{\defcomunidade}{Def.1 \ (a) \text{,} \ (b) \ \text{e} \ (c)}

\newcommand{\newbegin}{\vspace{0.5cm}}

\newcommand{\newl}{\vspace{0.1cm}}
\newcommand{\escolhe}[2]{\resizebox{!}{15pt}{$\displaystyle \binom{#1}{#2}$}}

\newtheorem{definicao}{Def} 
\newtheorem{exemplo}{Exemplo}  
\newtheorem{lema}{Lema} 
\newtheorem{teo}{Teorema}  
\newtheorem{cor}{Corolário}
\newtheorem{prop}{Proposição}
\newtheorem{defi}{Definição}
 
% PARTE DOIS 
% PARTE DOIS 
% PARTE DOIS 
% PARTE DOIS 

{
 
\begin{defi}
	\label{defi1}
	Seja $\mathcal{P} = \{P_1, P_2, \ldots P_k \}$ uma $k-$partição de $d$ elementos distintos. Dizemos que $\mathcal{P}$ é uma $k-$partição par se $|P_i|$ é par, para todo $P_i \in \mathcal{P}$.
\end{defi} \newbegin
 

\begin{defi}
	\label{defi2}
	Denotamos por $\mu(d, k)$ a quantidade de $k-$partições pares distintas de $d$ elementos distintos.
\end{defi} \newbegin

\begin{defi}
	\label{def3}
	Denotamos por $\varphi_2(d, k)$ a quantidade de $k-$partições $\mathcal{P}$ não pares de $d$ elementos onde somente as duas primeiras partes $P_1, P_2 \in \mathcal{P}$ possuem cardinalidade ímpar. 
\end{defi} \newbegin

\begin{defi}
	\label{def4}
	Seja $\mathcal{H}$ um hipergrafo e seja $C : V(\mathcal{H}) \rightarrow [k]$ uma $k$-coloração dos vértices de $\mathcal{H}$. Dizemos que $C$ é uma $k$-coloração ímpar se, para toda aresta $e \in E(\mathcal{H})$, existe uma cor que aparece impar vezes nos vértices de $e$. Denotamos por $\chi_{io}(\mathcal{H})$ o menor inteiro $k$ tal que  $\mathcal{H}$ possui uma $k-$coloração ímpar.
	
\end{defi} \newbegin

\begin{defi}
	\label{def5}
	Seja $G$ um grafo e seja $C : V(\mathcal{H}) \rightarrow [k]$ uma $k$-coloração dos vértices de $G$. Dizemos que $C$ é uma $k$-coloração ímpar se, para todo vértice $v \in V(G)$, existe uma cor que aparece impar vezes na vizinhança de $G$. Denotamos por $\chi_{o}(\mathcal{H})$ (resp. $\chi_{io}(\mathcal{H})$) o menor inteiro $k$ tal que $G$ possui uma $k-$coloração ímpar própria (resp. não própria).
\end{defi} \newbegin


\begin{prop}
	\label{prop1}
	Seja $2n!!$ o fatorial dos ímpares. Temos que $2n!! \leq n^n$.
\end{prop} \newpage

\begin{teo}
	\label{teo1} 
	Seja $\mathcal{H}$ um hipergrafo tal que cada aresta $e \in E(\mathcal{H})$ possui pelo menos $2t$ vértices, para $t > 0$, e cada aresta $e$ intersecta no máximo $\Gamma$ outras arestas. Temos que $\chi_{io}(\mathcal{H}) \leq t {\cdot} (e {\cdot} (\Gamma + 1))^{1/t}$
\end{teo} \newbegin
 

\begin{cor} 
	\label{cor1}  
	$\chi_{io}(\mathcal{H}) \in \mathcal{O}(ln(\Gamma) {\cdot} \Gamma^{1/t})$
\end{cor}

	A demonstração segue do Teorema $\ref{teo1}$. Se $t \geq 1 + ln (\Gamma + 1)$, substituindo $t$ por $ln(e {\cdot} (\Gamma + 1))$, temos que $\chi_{io}(\mathcal{H}) \in \mathcal{O}(ln(\Gamma))$. Se $t < 1 + ln (\Gamma + 1)$, então $\chi_{io}(\mathcal{H}) \leq (1 + ln (\Gamma + 1)) {\cdot} (e {\cdot} (\Gamma + 1))^{1/t}$, pelo Teorema $\ref{teo1}$, e o resultado segue. \newbegin
	
\begin{cor}
	\label{cor2}  
	Seja $G$ um grafo com $\delta(G) \geq 2t$, para $t > 0$. Temos que $\chi_{io}(G) \leq t {\cdot} (e {\cdot} (\Delta^2 + 1))^{1/t}$.
\end{cor}
	 
	  Seja $\mathcal{H} = (V(G), E)$ um hipergrafo tal que $E = \bigcup\limits_{v \in V(G)}N_G(v)$, \ie, $\mathcal{H}$ possui uma aresta $e_v$ para cada vértice $v \in V(G)$ e cada aresta $e_v \in E(\mathcal{H})$ contém os vértices adjacentes a $v$ em $G$. Note que uma $k$-coloração  ímpar $c$ para $\mathcal{H}$ também é uma $k$-coloração ímpar para $G$. Como cada aresta em $\mathcal{H}$ intersecta no máximo $\Delta^2$ outras arestas, pelo Teorema $\ref{teo1}$, temos o resultado desejado. \newbegin

\begin{cor}
	\label{cor3}  
	 $\chi_{io}(G) \in \mathcal{O}(ln(\Delta) {\cdot} \Delta^{2/t})$.
\end{cor}
     
     A demonstração é similar à demonstração do Corolário $\ref{cor1}$.
\newbegin

\begin{teo}
	\label{teo2}  
	Seja $G$ um grafo com $\delta(G) \geq 2t$, para $t > 0$. Se $\chi(G) \in \mathcal{O}(1)$, então $\chi_{o}(G) \in \mathcal{O}(ln(\Delta) {\cdot} \Delta^{2/t})$ .
\end{teo} \newbegin

\begin{teo}
	\label{teo3}   
	Sejam $d$ e $k$ inteiros positivos, com $d$ par. Temos que $ \sum\limits_{i = 0}^{ k }i^{d}   \leq k^{d-t} {\cdot} t^t  {\cdot} 2^{2k-t-1} $, para qualquer $1 \leq t \leq \dfrac{d}{2}$.
\end{teo} \newbegin

\begin{cor}
	Sejam $d$ e $k$ inteiros positivos, com $d$ ímpar. Temos que $ \sum\limits_{i = 0}^{ k }i^{d}   \leq k^{d+1-t} {\cdot} t^t  {\cdot} 2^{2k-t-1} $, para qualquer $1 \leq t \leq \blceil \dfrac{d}{2} \brceil$.
\end{cor}

Pelo Teorema $\ref{teo3}$, temos que $ \sum\limits_{i = 0}^{ k }i^{d+1}   \leq k^{d+1-t} {\cdot} t^t  {\cdot} 2^{2k-t-1} $, para qualquer $1 \leq t \leq \dfrac{d+1}{2}$. Como $ \sum\limits_{i = 0}^{ k }i^{d} \leq  \sum\limits_{i = 0}^{ k }i^{d+1}$, temos o resultado desejado.

 \newpage
 
 \section{Demonstração do Teorema $\ref{teo1}$} \newl
 
 \begin{lema}  
 	\label{lema1} 
 	$\mu(2d, k) \leq (2d - 1) \cdot k \cdot \mu(2d - 2, k)$, para $d \geq 1$.
 \end{lema}
 
 \begin{proof} Iremos construir as $k$-partições pares $\mathcal{P}$ possíveis de $2d$ elementos com base nas escolhas que temos para um determinado elemento $x \in [2d]$. Devemos escolher uma parte $P_i$ para $x$ pertencer e temos $k$ partes disponíveis para $x$. Como cada parte tem tamanho par, devemos escolher um elemento $y \in [2d]$ diferente de $x$ para pertencer também à parte $P_i$. Temos $2d - 1$ escolhas para este caso. Por fim, devemos particionar os $2d - 2$ elementos restantes em $k$ partes de modo que cada parte tenha tamanho par. Sendo assim, devemos escolher uma $k$-partição par $\mathcal{P'}$ de $2d - 2$ elementos. Logo, $\mu(2d, k) \leq (2d - 1) \cdot k \cdot \mu(2d - 2, k)$. 
 	
 \end{proof} \newl
 
 
  \begin{lema}  
 	\label{lema2} 
 	$\mu(2d, k) \leq (d {\cdot} k)^d$, para $d \geq 1$.
 \end{lema}
 
 \begin{proof} Iremos demonstrar por indução em $d$ que $\mu(2d, k) \leq 2d!! \cdot k^d$. Como $2d!! \leq d^d$, pela Proposição $\ref{prop1}$, disto segue que $\mu(2d, k) \leq (d {\cdot} k)^d$. 
 	
 	Base ($d = 1$): Pelo Lema $\ref{lema1}$, temos que $\mu(2, k) \leq k \cdot \mu(0, k)= 2!! \cdot k$ e o resultado segue.
 	
 	Passo ($d > 1$): Suponha que $\mu(2\ell, k) \leq (2\ell)!! \cdot k^{\ell}$, para $1 \leq \ell < d$. Pelo Lema $\ref{lema1}$, $\mu(2d, k) \leq (2d - 1) \cdot k \cdot \mu(2 {\cdot} (d - 1), k)$. Por $HI$, temos que: 
 	\begin{align}
 		\begin{split}
			\mu(2 {\cdot} (d - 1), k) \leq (2d - 2)!! \cdot k^{d-1}
 		\end{split} 
 	\end{align} 
 	
 	Portanto: 
 	\begin{align}
 		\begin{split}
 			\mu(2d, k) &\leq (2d - 1) \cdot k \cdot \mu(2 {\cdot} (d - 1), k) \\
 			&\leq (2d - 1) \cdot k \cdot (2d - 2)!! \cdot k^{d-1} \\
 			&\leq 2d!! \cdot k^d
 		\end{split} 
 	\end{align} 
 	
 \end{proof} \newl
 
 
\begin{lema}   
	\label{lema3}
	\begin{equation}
		\mu(2d, k) =
		\begin{cases}
			1 & \text{$k = 1$} \\
			\sum\limits_{i = 0}^{d} \escolhe{2d}{2i} {\cdot} \mu(2i, k-1) & \text{c.c.} \\ 
		\end{cases}
	\end{equation} 
\end{lema}
 
\begin{proof} 
	Se $k=1$, então $\mu(2d, k) = 1$, pois os $2d$ elementos devem estar contidos em uma única parte. Sendo assim, considere que $k > 1$. Agora, iremos construir as $k$-partições pares $\mathcal{P}=\{P_1, P_2, \ldots P_k\}$ possíveis de $2d$ elementos. Primeiro, devemos escolher quantos dos $2d$ elementos irão pertencer a parte $P_k$. Como $|P_k|$ é par, podemos escolher qualquer inteiro $i$ entre $0$ e $d$, de modo que $|P_k| = 2i$. Como os $2d$ elementos são distintos, temos $\escolhe{2d}{2i} = \escolhe{2d}{2d - 2i}$ maneiras de escolher $2i$ elementos para a parte $P_k$. Após isso, devemos particionar os $2d - 2i$ elementos restantes em $(k-1)$ partes de tamanho par, sendo assim, devemos escolher uma $(k-1)$-partição par $\mathcal{P'}$ de $2d - 2i$ elementos. Portanto: 
	
	\begin{align}
		\begin{split}
			\mu(2d, k) &= \sum\limits_{i = 0}^{d}\escolhe{2d}{2i} {\cdot} \mu(2d - 2i, k-1) \\
			&= \sum\limits_{i = 0}^{d}\escolhe{2d}{2d - 2i} {\cdot} \mu(2d - 2i, k-1)\\
			&= \sum\limits_{i = 0}^{d}\escolhe{2d}{2i} {\cdot} \mu(2i, k-1)  
		\end{split} 
	\end{align}  
\end{proof} \newl


\begin{lema}    
	\label{lema4}
	\begin{equation}
		\varphi_2(2d, k) =
		\begin{cases}
			0 & \text{se $k \leq 1$}\\
			2^{2d - 1} & \text{se $k = 2$} \\
			\sum\limits_{i = 1}^{d} \escolhe{2d}{2i} {\cdot} \varphi_2(2i, k-1) & \text{c. c.} \\ 
		\end{cases}
	\end{equation} 
\end{lema}

\begin{proof} 
	Iremos analisar cada caso da recorrência separadamente. \newl
	
	Caso 1 ($k \leq 1$): Se $k \leq 1$, então não há como $k$-particionar os $2d$ elementos de modo que apenas as partes $P_1$ e $P_2$ tenham tamanho ímpar. Portanto, $\varphi(2d, k) = 0$. \newl
	
	Caso 2 ($k = 2$): Se $k = 2$, então qualquer $k$-partição $\mathcal{P} = \{P_1, P_2\}$ de $2d$ elementos é contada em $\mu(2d, 2)$ ou em $\varphi_2(2d, 2)$, pois como temos um número par de elementos, ambas partes $P_1$ e $P_2$ têm tamanho par ou ímpar. Logo $\mu(2d, 2) + \varphi_2(2d, 2) = 2^{2d}$, pois $2^{2d}$ é o total de $2$-partições possíveis de $2d$ elementos. Pelo Lema $\ref{lema3}$:
	
	\begin{align}
		\begin{split}
			\mu(2d, 2) &= \sum\limits_{i = 0}^{d}\escolhe{2d}{2i} {\cdot} \mu(2i, 1) = \sum\limits_{i = 0}^{d}\escolhe{2d}{2i} = 2^{2d-1}
		\end{split} 
	\end{align} 
	
	Logo, temos que $\varphi_2(2d, 2)  = 2^{2d} - 2^{2d - 1} = 2^{2d - 1}$ e o resultado segue. \newl
	
	
	Caso 3 ($k > 2$): Iremos construir as $k$-partições não pares $\mathcal{P} = \{P_1, P_2, \ldots P_k\}$ possíveis de $2d$ elementos. Como $k > 2$, existe uma parte $P_i \in \mathcal{P}$, onde $|P_i|$ é par. Sendo assim, a demonstração segue de modo análogo à demonstração do Lema $\ref{lema3}$, com a única restrição de que $|P_i| < 2d$, pois as partes $P_1$ e $P_2$ tem ao menos um elemento.
\end{proof} \newl
  

\begin{lema} 
	\label{lema5} 
	$\mu(2d, k) \leq \mu(2d - 2, k) {\cdot} k^{2}$, para $d \geq 1$.
\end{lema}

\begin{proof}
	Iremos provar que $\mu(2d, k) = k {\cdot} \mu(2d - 2, k) + (k^2 - k) {\cdot} \varphi_2(2d - 2, k)$. Disto segue que $\mu(2d, k) \leq \mu(2d - 2, k) {\cdot} k^{2}$, pois, pelos Lemas $\ref{lema3}$ e $\ref{lema4}$, temos que $\varphi_2(2d - 2, k) \leq \mu(2d - 2, k)$, considerando que ambos possuem uma recorrência similar e ainda $\varphi_2(2d, 2) = \mu(2d, 2)$. Sejam dois elementos distintos $x, y \in [2d]$. Iremos construir uma $k$-partição par $\mathcal{P}$ de $2d$ elementos com base em duas escolhas: se $x$ e $y$ irão pertencer a mesma parte $P_i \in \mathcal{P}$ ou não. \newl
	
	Caso 1: Se escolhermos que $x$ e $y$ irão pertencer a mesma parte $P_i \in \mathcal{P}$, então devemos escolher uma parte $P_i$ das $k$ partes disponíveis. Após, devemos escolher uma $k$-partição par $\mathcal{P'}$ de $2d - 2$ elementos para os elementos restantes. Sendo assim, para este caso, temos $k {\cdot} \mu(2d - 2, k)$ partições possíveis. \newl 
	
	Caso 2: Se escolhermos que $x \in P_i$ e $y \in P_j$, onde $P_i \neq P_j$, então devemos escolher primeiro quais são as partes $P_i$ e $P_j$ dentre as $k$ partes que irão conter $x$ e $y$ respectivamente. Temos $2 {\cdot} \escolhe{k}{2} = k^2 - k$ formas de escolher $P_i$ e $P_j$, pois há $\escolhe{k}{2}$ maneiras de escolher duas das $k$ partes disponíveis e há duas maneiras de escolher qual das duas partes irá conter cada elemento. Após, devemos escolher uma $k$-partição não par $\mathcal{P'}$ onde apenas as partes $P_i$ e $P_j$ tenham tamanho ímpar. Por simetria das partes, há exatamente $\varphi_2(2d - 2, k)$ partições $\mathcal{P'}$ distintas. Logo, para este caso, temos $(k^2 - k) {\cdot} \varphi_2(2d - 2, k)$ partições possíveis.
	
	
\end{proof} \newl

 
\begin{proof}[Demonstração do Teorema~\ref{teo1}]
	Pinte os vértices de $\mathcal{H}$ com $k = t {\cdot} (e {\cdot} (\Gamma + 1))^{1/t}$ cores aleatoriamente e independentemente. Seja $X_{e}$ o evento da aresta $e \in E(\mathcal{H})$ não ter uma cor que apareça ímpar vezes, onde $|e|$ é ímpar. Note que $\mathds{P}[X_{e}] = \dfrac{\mu(|e|, k)}{k^{|e|}}$. Como $|e| \geq 2t$, pelo Lema $\ref{lema5}$, temos que: 
	\begin{align}
		\begin{split}
			\mu(|e|, k) \leq \mu(|e| - (|e| - 2t), k) {\cdot} k^{|e| - 2t} = \mu(2t, k) {\cdot} k^{|e| - 2t}
		\end{split} 
	\end{align} 

	Pelo Lema $\ref{lema2}$:
 	\begin{align}
 		\begin{split}
 			\mu(2t, k) \leq (t {\cdot } k)^t
 		\end{split} 
 	\end{align} 
  
    Portanto, por $(7)$ e $(8)$:
    
	\begin{align}
		\begin{split}
			\mathds{P}[X_{e}] = \dfrac{\mu(|e|, k)}{k^{|e|}} 
			 \leq \left(\dfrac{t}{k}\right)^t  = \left(\dfrac{t}{t {\cdot}   (e {\cdot} (\Gamma + 1))^{1/t}}\right)^t = \dfrac{1}{e {\cdot} (\Gamma + 1)}
		\end{split} 
	\end{align}
	
  Sendo assim, pelo Lema Local de Lovász, temos que $\mathds{P}[\bigcap\limits_{e \in E(\mathcal{H})} \overline{X_{e}} ] > 0$. Portanto, existe uma $k$-coloração tal que existe uma cor que aparece ímpar vezes nos vértices de $e$, para toda aresta $e \in E(\mathcal{H})$. Logo, $\chi_{io}(\mathcal{H}) \leq k = t {\cdot} (e {\cdot} (\Gamma + 1))^{1/t}$.
	
\end{proof}\newbegin
 
  \section{Demonstração do Teorema $\ref{teo2}$} \newl
  
 \begin{proof}[Demonstração do Teorema~\ref{teo2}]
 	Iremos provar que $\chi_{o}(G) \leq \chi(G) {\cdot} \chi_{io}(G)$. Disto segue que $\chi_{io}(G) \in \mathcal{O}(ln(\Delta) {\cdot} \Delta^{2/t})$, pois $\chi(G) \in \mathcal{O}(1)$ e, pelo Corolário $\ref{cor3}$, $\chi_{io}(G) \in \mathcal{O}(ln(\Delta) {\cdot} \Delta^{2/t})$. Seja $\mathcal{P} = \{P_1, P_2 \ldots, P_k\}$ uma $k$-coloração ímpar mínima. Para cada parte $P_i \in \mathcal{P}$, pinte os vértices de $P_i$ com uma coloração própria mínima de $G[P_i]$, utilizando cores distintas para cada parte $P_i \in \mathcal{P}$. Ao final, temos uma coloração $c$ de $G$ utilizando no máximo $\chi(G) {\cdot} \chi_{io}(G)$ cores. Afirmamos que $c$ é uma coloração ímpar e própria. Note que $c$ é própria, pois tome $uv \in E(G)$, se $u \in P_i$ e $v \in P_j$, tal que $i \neq j$, onde $P_i, P_j \in \mathcal{P}$, então $u$ e $v$ possuem cores distintas, se $u, v \in P_i$, então $u$ e $v$ possuem cores distintas, pois $P_i$ é colorido propriamente com $\chi(G[P_i])$ cores. Observe que, para todo $v \in V(G)$, existe uma parte $P_i \in \mathcal{P}$, tal que $|S|$ é ímpar, onde $S = P_i \cap N_G(v)$. Seja $T = N_G(v) \setminus S$. Note que existe uma cor em $c$ que aparece ímpar vezes em $S$, pois $|S|$ é ímpar. Como $c(s) \neq c(t)$, para $s \in S$ e $t \in T$, pois $s$ e $t$ pertencem a partes distintas de $\mathcal{P}$, temos que existe uma cor que aparece ímpar vezes em $N_G(v)$. 
 	
 \end{proof}
 
  \section{Demonstração do Teorema $\ref{teo3}$} \newl
 
 $\textbf{Lembrete:}$ \newl
 
 $e^x = \sum\limits_{i = 0}^{\infty}\dfrac{x^i}{i!} = x^0 + \dfrac{x^1}{1!} + \dfrac{x^2}{2!} + \dfrac{x^3}{3!} + \ldots$ \newl
 
  $e^{-x} = \sum\limits_{i = 0}^{\infty}(-1)^i {\cdot } \dfrac{ x^i}{i!} = x^0 +-\dfrac{x^1}{1!} + \dfrac{x^2}{2!} - \dfrac{x^3}{3!} + \ldots$ \newl
 
   $\dfrac{e^x + e^{-x}}{2} = \sum\limits_{i = 0}^{\infty}  \dfrac{ x^{2}}{2i!} = x^0 +-\dfrac{x^2}{2!} + \dfrac{x^4}{4!} - \dfrac{x^6}{6!} + \ldots$ \newl
   
  \vspace{5mm}
  
 \begin{lema}
 	\label{lema6}
 	$\mu(d, k) = \dfrac{1}{2^k} {\cdot} \sum\limits_{i = 0}^{k} \escolhe{k}{i} {\cdot} (k - 2i)^{d}$
 \end{lema}
 
 \begin{proof} 
 	Note que: \newl
 	
 	\begin{align}
 		\begin{split}
 			\mu(d, k) =  \mathop{\sum_{a_1 + a_2 + \ldots a_k = d}}_{\text{com $a_i$ par p/ $1 \leq i \leq k$}}\binom{d}{a_1, a_2, \ldots, a_k} 
 		\end{split} 
 	\end{align}
 	
 	Ou seja, $\mu(d, k)$ é igual ao somatório multinomial de todas as possibilidades onde $a_1 + a_2 + \ldots a_k = d$ com $a_i$ par, para todo $1 \leq i \leq k$. Sendo assim:
 	
 	\begin{align}
 		\begin{split}
 			\label{l6_1}
 			\mu(d, k) &=  \mathop{\sum_{a_1 + a_2 + \ldots a_k = d}}_{\text{com $a_i$ par p/ $1 \leq i \leq k$}}\binom{d}{a_1, a_2, \ldots, a_k} \\
 			&= \mathop{\sum_{a_1 + a_2 + \ldots a_k = d}}_{\text{com $a_i$ par p/ $1 \leq i \leq k$}}\dfrac{d!}{a_1! {\cdot} a_2! \ldots {\cdot} a_k!} \\
 			\therefore \dfrac{\mu(d, k)}{d!} &= \mathop{\sum_{a_1 + a_2 + \ldots a_k = 2}}_{\text{com $a_i$ par p/ $1 \leq i \leq k$}}\dfrac{1}{a_1! {\cdot} a_2! \ldots {\cdot} a_k!}
 		\end{split} 
 	\end{align}

 
 Note que $\sum\limits_{a_1 + a_2 + \ldots a_k = d \atop \text{com $a_i$ par p/ $1 \leq i \leq k$}} \dfrac{1}{a_1! {\cdot} a_2! \ldots {\cdot} a_k!}$ é o coeficiente de $x^{d}$ na série de potência $\left(\dfrac{e^x + e^{-x}}{2}\right)^k$. Sendo assim, agora iremos calcular qual o valor do coeficiente $x^{d}$ nesta série. Note que:
 
 \begin{align}
 	\begin{split}
 		\label{l6_2}
 		\left(\dfrac{e^x + e^{-x}}{2}\right)^k &= \dfrac{1}{2^k} {\cdot} (e^x + e^{-x})^k \\
 		&= \dfrac{1}{2^k} {\cdot} \sum\limits_{i = 0}^{k}\escolhe{k}{i}(e^x)^{k-i} {\cdot} (e^{-x})^i \\
 		&= \dfrac{1}{2^k} {\cdot} \sum\limits_{i = 0}^{k}\escolhe{k}{i}(e^x)^{k-i} {\cdot} (e^{x})^{-i} \\
 		&= \dfrac{1}{2^k} {\cdot} \sum\limits_{i = 0}^{k}\escolhe{k}{i}(e^x)^{k-2i}  \\
 		&= \dfrac{1}{2^k} {\cdot} \sum\limits_{i = 0}^{k}\escolhe{k}{i}e^{x {\cdot}(k - 2i)}
 	\end{split} 
 \end{align}
 
 Observe que:
 
  \begin{align}
 	\begin{split}
 		\label{l6_3}
 		e^{x {\cdot}(k - 2i)} = \sum\limits_{j = 0}^{\infty} \dfrac{(x {\cdot}(k - 2i))^j}{j!} = \sum\limits_{j = 0}^{\infty} \dfrac{x^j {\cdot}(k - 2i)^j}{j!}
 	\end{split} 
 \end{align}
 
 Por $(\ref{l6_1})$, $(\ref{l6_2})$ e $(\ref{l6_3})$, temos que:
 
  \begin{align}
 	\begin{split} 
 		\sum\limits_{a_1 + a_2 + \ldots a_k = d \atop \text{com $a_i$ par p/ $1 \leq i \leq k$}} \dfrac{1}{a_1! {\cdot} a_2! \ldots {\cdot} a_k!} &= \dfrac{1}{2^k} {\cdot} \sum\limits_{i = 0}^{k}\escolhe{k}{i} {\cdot} \dfrac{(k - 2i)^{d}}{d!} \\
 		\therefore \mu(d, k) &= \dfrac{1}{2^k} {\cdot} \sum\limits_{i = 0}^{k}\escolhe{k}{i} {\cdot} (k - 2i)^{d}
 	\end{split} 
 \end{align}
 
 \end{proof}
 
 \begin{proof}[Demonstração do Teorema~\ref{teo3}]
 	Pelo Lema $\ref{lema6}$:
 	
 	\begin{align}
 		\begin{split} 
 			 \mu(d, k) &= \dfrac{1}{2^k} {\cdot} \sum\limits_{i = 0}^{k}\escolhe{k}{i} {\cdot} (k - 2i)^{d} \\
 			 &= \dfrac{1}{2^k} {\cdot} \left(  \sum\limits_{i = 0}^{ \blfloor \frac{k}{2} \brfloor  }\escolhe{k}{i} {\cdot} (k - 2i)^{d} +   \sum\limits_{i = \blfloor \frac{k}{2} \brfloor + 1  }^{k}\escolhe{k}{i} {\cdot} (k - 2i)^{d} \right)
 		\end{split} 
 	\end{align}
 	
 	Sejam $x = \sum\limits_{i = 0}^{ \blfloor \frac{k}{2} \brfloor  }\escolhe{k}{i} {\cdot} (k - 2i)^{d} $ e $y = \sum\limits_{i = \blfloor \frac{k}{2} \brfloor + 1  }^{k}\escolhe{k}{i} {\cdot} (k - 2i)^{d}$. Iremos provar que $x=y$. Para isso, suponha que $k$ é par. Temos que: 
 	\begin{align}
 		\begin{split}  
 			x = \escolhe{k}{0}k^{d} + \escolhe{k}{1}(k-2)^{d} + \escolhe{k}{2}(k-4)^{d} \ldots + \escolhe{k}{\frac{k}{2} - 1}2^{d} + \escolhe{k}{\frac{k}{2}}0^{d}
 		\end{split} 
 	\end{align}
 	
 	Note que: 
 	\begin{align}
 		\begin{split}  
 			y &= \escolhe{k}{\frac{k}{2} + 1}(-2)^{d} + \escolhe{k}{\frac{k}{2} + 2}(-4)^{d} \ldots + \escolhe{k}{k-1}(-(k-2))^{d} + \escolhe{k}{k}(-k)^{d} \\
 			&= \escolhe{k}{\frac{k}{2} - 1}2^{d} + \escolhe{k}{\frac{k}{2} - 2}4^{d} \ldots + \escolhe{k}{1}(k-2)^{d} + \escolhe{k}{0}k^{d}
 		\end{split} 
 	\end{align}
 	
 	Logo $x = y$. Sendo assim:
 	
 	\begin{align}
 		\begin{split}
 			\label{l6_4} 
 			\mu(d, k) &= \dfrac{2}{2^k} {\cdot}  \sum\limits_{i = 0}^{  \frac{k}{2}   }\escolhe{k}{i} {\cdot} (k - 2i)^{d}  \\
 			&= \dfrac{1}{2^{k-1}} {\cdot}  \sum\limits_{i = 0}^{  \frac{k}{2}   }\escolhe{k}{i} {\cdot} (k - 2i)^{d} \\
 			&\geq \dfrac{1}{2^{k-1}} {\cdot}  \sum\limits_{i = 0}^{  \frac{k}{2}   }(k - 2i)^{d} \\
 			&\geq \dfrac{1}{2^{k-1}} {\cdot}  \sum\limits_{i = 0}^{  \frac{k}{2}   }2i^{d} \\
 			&\geq \dfrac{2^{d}}{2^{k-1}} {\cdot}  \sum\limits_{i = 0}^{  \frac{k}{2}   }i^{d}
 		\end{split} 
 	\end{align}
 	 
 	Pelos Lema $\ref{lema2}$ e $\ref{lema5}$:
 	
 	 \begin{align}
 		\begin{split} 
 			\label{l6_5} 
 			\mu(d, k) \leq \mu(2t, k) {\cdot} k^{d - 2t}  \leq  (t {\cdot} k)^{t} {\cdot} k^{d - 2t} \leq t^t {\cdot} k^{d-t}, \ \text{para} \ 1 \leq t \leq \frac{d}{2}
 		\end{split} 
 	\end{align}
 	
 	Por ($\ref{l6_4}$) e ($\ref{l6_5}$), temos que:
 	
 	\begin{align}
 		\begin{split} 
 			 \sum\limits_{i = 0}^{  \frac{k}{2}   }i^{d} &\leq \left(\dfrac{k}{2} \right)^{d} {\cdot} \left(\dfrac{t}{k} \right)^t  {\cdot} 2^{k-1}  \\
 			  (\rightarrow) \ \ \sum\limits_{i = 0}^{ k }i^{d} &\leq  k^{d} {\cdot} \left(\dfrac{t}{2k} \right)^t  {\cdot} 2^{2k-1}    \\ 
 			  (\rightarrow) \ \ \sum\limits_{i = 0}^{ k }i^{d}  &\leq k^{d-t} {\cdot} t^t  {\cdot} 2^{2k-t-1} 
 		\end{split} 
 	\end{align}
  	
\end{proof}
\end{document}
 