\documentclass[12pt]{article}
\usepackage{cancel}
\usepackage{sbc-template/sbc-template}
\usepackage{graphicx,url}
\usepackage[utf8]{inputenc}
\usepackage[brazil]{babel}    
\usepackage{geometry}
\usepackage{setspace} 
\usepackage[T1]{fontenc}
\usepackage[brazil]{babel}
\usepackage{amsthm}
\usepackage[portuguese, ruled, lined, linesnumbered, commentsnumbered, longend]{algorithm2e}
\usepackage{blindtext}
\usepackage{geometry}
\usepackage{scrextend}
\usepackage{setspace}
\usepackage{amssymb}
\usepackage{amsmath}
\usepackage{dsfont}
\usepackage{comment}
 
\newcommand{\mycomment}[1]{}
\newenvironment{citacao}{%
\begin{list}{}{%
\setlength{\topsep}{1pt}%
\setlength{\leftmargin}{4cm}%
\setlength{\rightmargin}{0cm}%
\setlength{\listparindent}{\parindent}%
\setlength{\itemindent}{\parindent}%
\setlength{\parsep}{\parskip}%
}%
\item[]}{\end{list}}

%\renewcommand{\baselinestretch}{0.6} 

\sloppy
\raggedbottom
 
\begin{document} 
 \def\changemargin#1#2{\list{}{\rightmargin#2\leftmargin#1}\item[]} 
\let\endchangemargin=\endlist 
\newcommand{\wbigcup}{\mathop{\widetilde{\bigcup}}\displaylimits}

\newcommand{\overbar}[1]{\mkern 1.5mu\overline{\mkern-1.5mu#1\mkern-1.5mu}\mkern 1.5mu}
\newcommand{\ie}{\textit{i}.\textit{e}.} 
\newcommand{\blceil}{\left \lceil}
\newcommand{\brceil}{\right \rceil} 
\newcommand{\blfloor}{\left \lfloor}
\newcommand{\brfloor}{\right \rfloor} 
\newcommand{\defcomunidade}{Def.1 \ (a) \text{,} \ (b) \ \text{e} \ (c)}

\newcommand{\newbegin}{\vspace{0.5cm}}

\newcommand{\newl}{\vspace{0.1cm}}
\newcommand{\escolhe}[2]{\resizebox{!}{15pt}{$\displaystyle \binom{#1}{#2}$}}

\newtheorem{definicao}{Def} 
\newtheorem{exemplo}{Exemplo}  
\newtheorem{lema}{Lema} 
\newtheorem{teo}{Teorema}  
\newtheorem{cor}{Corolário}
\newtheorem{prop}{Proposição}
\newtheorem{defi}{Definição}

% PARTE UM 
% PARTE UM 
% PARTE UM 
% PARTE UM 

{
\begin{prop}
	\label{prop1}
	Seja $G$ um grafo conexo. Seja $T$ uma árvore de Busca em Largura de $G$ a partir de um vértice $v$ qualquer. Se existem vértices $s, t \in V(G)$ tais que $st \in E(G)$, $st \notin E(T)$ e $dist_T(v, s) = dist_T(v, t) + 1$, então $G$ possui ciclo par.
\end{prop} \newbegin

\begin{prop}
	\label{prop2}
	Seja $G$ um grafo conexo livre de ciclos pares. Seja $T$ uma Árvore de Busca em Largura de $G$ a partir de $v \in V(G)$. Seja $V \subsetneq V(G)$ o conjunto de vértices com distância $p > 0$ de $v$. Temos que o conjunto $E(G[V])$ é um emparelhamento.
\end{prop} \newbegin


\begin{lema}  
	\label{lema1}
	Seja $G$ um grafo. Se existe uma $k$-partição $V_1, V_2, \ldots ,V_k$ dos vértices de $G$ tal que, para todo vértice $v \in V(G)$, temos que $|N(v) \cap V_i| = 1$, para algum $1 \leq i \leq k$, então $\chi_{pcf}(G) \leq  \sum\limits_{i = 1}^{k} \chi(G[V_i])$.
\end{lema}

\begin{proof}
	
	Seja $H_i = G[V_i]$, para todo $1 \leq i \leq k$. Iremos colorir cada subgrafo $H_i$ com $\chi(H_i)$ cores distintas. Para isso, cada cor será representada por um par ordenado. Seja $c_i: V(H_i) \rightarrow \{ i \} \times \chi(H_i)$ uma coloração própria de $H_i$. Para todo par distinto de colorações $c_i$ e $c_j$, temos que $c_i(v) \neq c_j(u)$, para todo $ v \in V(H_i)$ e $ u \in V(H_j)$, pois $(i, x) \neq (j, y)$ para $i \neq j$.\newl
	
	Seja $c$ uma coloração de $G$ tal que $c(v) = c_i(v)$ se e somente se $v \in V(H_i)$. Em outras palavras, $c$ é a união das colorações usadas em cada subgrafo $H_i$. Como $c_i$ é uma coloração própria de $H_i$ e todo par distinto de subgrafos $H_i$ e $H_j$ são coloridos com cores distintas, temos que $c$ é uma coloração própria de $G$. \newl
	
	Como, para todo vértice $v \in V(G)$, vale que $|N(v) \cap V(H_i)| = 1$, para algum subgrafo $H_i$, e como as cores usadas em $H_i$ são distintas das cores usadas em $V(G)\setminus V(H_i)$, temos que existe uma cor $(i, x)$ que aparece uma única vez na vizinhança de $v$, para $x \in [\chi(H_i)]$. Sendo assim, $c$ descreve uma coloração própria livre de conflitos de $G$. Como $c_i$ utiliza $\chi(H_i)$ cores, para todo $1 \leq i \leq k$, temos que $\chi_{pcf}(G) \leq \sum\limits_{i = 1}^{k} \chi(H_i)$. 
	
\end{proof} 


\begin{teo}
	\label{teo1}
	Seja $G$ um grafo conexo. Se $G$ é livre de ciclos pares, então $\chi_{pcf}(G) \leq 7$.
\end{teo}

\begin{proof}
	Seja $T$ uma Árvore de Busca em Largura de $G$ a partir de um vértice $r$ qualquer. Sabemos que $T$ é uma árvore geradora, pois $G$ é conexo. Seja $V_0, V_1$, $V_2$ uma partição de $G$ tal que $x \in V_i$ se e somente se $i = dist_T(r, x) \pmod{3} $. Seja $s$ um vértice de $G$, tal que $s \neq r$. Seja $p$ o pai de $s$ em $T$ e seja $f$ um filho de $s$ em $T$.
	Note que $p$ e $f$ pertencem a partições distintas, pois:
	\begin{align}
		\begin{split}
			dist_T(r, p) \pmod{3} \neq dist_T(r, p) + 2 \pmod{3} = dist_T(r, f) \pmod{3}
		\end{split} 
	\end{align} 
	
	Seja $t \in V(G)$ um vértice tal que $dist_T(r, s) > dist_T(r, t) + 1$. Sabemos que $st \notin E(G)$, pois $T$ é uma árvore de Busca em Largura. Sendo assim, se $st \in E(G)$ e $st \notin E(T)$, então $dist_T(r, s) = dist_T(r, t) + 1$ ou $dist_T(r, s) = dist_T(r, t)$. Pela Proposição $\ref{prop1}$, sabemos que se $st \in E(G)$ e $st \notin E(T)$, então $dist_T(r, s) = dist_T(r, t)$, pois $G$ é livre de ciclos pares. Note que isto implica que $s$ é adjacente a precisamente um vértice $u$ em $G$ tal que $dist_T(r, s) = dist_T(r, u) + 1$, e, sendo assim, $u$ é o pai de $s$ em $T$, \ie, $u = p$. Note que $|N(s) \cap V_i| = 1$, onde $f \in V_i$, para todo $s \in V(G) \setminus\{r\}$. \newl
	
	Resta agora a partição do vértice raiz $r$. Iremos remover um vértice $v \in N(r)$ da partição $V_1$ e iremos construir uma nova partição $V_0, V^{'}_1, V_2, V_3$ de $G$, de modo que $V^{'}_1 = V_1\setminus\{v\}$ e $V_3 = \{v\}$. Seja $s$ um vértice onde $|N(s) \cap V_1| = 1$, \ie, o pai $p$ de $s$ pertence a $V_1$. Queremos argumentar que a propriedade é satisfeita para $s$ na nova partição $V_0, V^{'}_1, V_2, V_3$. Se $p \neq v$, então $|N(s) \cap V_1'| = 1$ e a propriedade continua valendo. Se $p = v$, então $N(s) \cap V_3 = \{v\}$, \ie, $ |N(s) \cap V_3| = 1$ e a propriedade vale. \newl
	
	Note que a partição $V_0, V_1', V_2 \text{ e } V_3$ satisfaz a condição do Lema $\ref{lema1}$. Note que pela Proposição $\ref{prop2}$, $E(G[V_i])$ é um emparelhamento. Sendo assim, temos que $\chi(G[V_i]) = 2$, para $0 \leq i \leq 2$. Note que $\chi(G[V_3]) = 1$, pois $V_3 = \{v\}$. Sendo assim, pelo Lema $\ref{lema1}$, temos que $\chi_{pcf}(G) \leq 7$.
	
\end{proof}  \newpage
}


% PARTE DOIS 
% PARTE DOIS 
% PARTE DOIS 
% PARTE DOIS 

{
 
\begin{defi}
	\label{defi1}
	Seja $\mathcal{P} = \{P_1, P_2, \ldots P_k \}$ uma $k-$partição de $d$ elementos distintos. Dizemos que $\mathcal{P}$ é uma $k-$partição par se $|P_i|$ é par, para todo $P_i \in \mathcal{P}$.
\end{defi} \newbegin
 

\begin{defi}
	\label{defi2}
	Denotamos por $\mu(d, k)$ a quantidade de $k-$partições pares distintas de $d$ elementos distintos.
\end{defi} \newbegin

\begin{defi}
	\label{def4}
	Denotamos por $\varphi_2(d, k)$ a quantidade de $k-$partições $\mathcal{P}$ não pares de $d$ elementos onde somente as duas primeiras partes $P_1, P_2 \in \mathcal{P}$ possuem cardinalidade ímpar. 
\end{defi} \newbegin

\begin{defi}
	\label{def5}
	Denotamos por $\chi_{io}(G)$ o menor inteiro $k$ tal que  $G$ possui uma $k-$coloração ímpar não própria. 
\end{defi} \newbegin

\begin{prop}
	\label{prop3}
	Seja $2n!!$ o fatorial dos ímpares. Temos que $2n!! \leq n^n$.
\end{prop} \newbegin
 
\begin{teo}
	\label{teo2} 
	$\chi_{io}(G) \leq \ell {\cdot} \sqrt[\ell]{e {\cdot} \Delta^2}$, para $1 \leq \ell \leq \blceil \dfrac{\delta(G)}{2} \brceil$. 
\end{teo}   

 Para demonstrar o Teorema $\ref{teo2}$, primeiro iremos demonstrar que $\mathds{P}[X_v] = \mathds{P}[Y_{d(v)}]$, onde $X_v$ é o evento do vértice $v \in V(G)$ não ter testemunha ímpar em uma $k$-coloração arbitrária e $Y_{d(v)}$ é o evento de uma $k$-partição $\mathcal{P}$ de $d(v)$ elementos ser par. Com isso, podemos analisar apenas $\mathds{P}[Y_{d(v)}] = \dfrac{\mu(d(v), k)}{k^{d(v)}}$. Como $\mu(d(v), k)$ é igual a $0$ para $d(v)$ ímpar, iremos considerar apenas quando $d(v)$ é par. Após, iremos demonstrar um limitante superior para $\mu(d(v), k)$ e, com isso, um limitante superior para $\mathds{P}[Y_{d(v)}]$. Por fim, iremos demonstrar que $\mathds{P}[Y_{d(u)}] \leq \mathds{P}[Y_{d(v)}]$, para $d(u) \geq d(v)$, \ie, a probabilidade de uma $k$-partição $\mathcal{P}$ ser par não aumenta conforme aumentamos o número de elementos que temos que particionar, considerando que $d(u)$ e $d(v)$ são pares. Isto implica que $\mathds{P}[X_v] = \mathds{P}[Y_{d(v)}] \leq \mathds{P}[Y_{2\ell}]$, para todo $v \in V(G)$ e $1 \leq \ell \leq \dfrac{\delta(G)}{2}$. Sendo assim, utilizando o Lema Local de Lovász, iremos limitar o problema da $k$-coloração ímpar não própria por $\delta(G)$.  \newpage 
 
 
 \begin{lema}  
 	\label{lema2} 
 	Seja $G$ um grafo de ordem $n$ colorido com $k$ cores uniforme e aleatoriamente. Seja $X_{v}$ o evento do vértice $v \in V(G)$ não ter testemunha ímpar. Seja $Y_{d(v)}$ o evento de uma $k-$partição de $d(v)$ elementos $\mathcal{P}$ ser par.  Temos que $\mathds{P}[X_{v}] = \mathds{P}[Y_{d(v)}]$.
 \end{lema}
 
 \begin{proof}
 	Sabemos que há $\mu(d(v), k)$ maneiras de $k-$colorir os vértices de $N(v)$, de modo que $v$ não possua testemunha ímpar. Note que temos exatamente $\mu(d(v), k) {\cdot} k^{n - d(v)}$ maneiras de colorir $G$ com $k$ cores de modo que $v$ não possua testemunha ímpar, pois ao colorir $N(v)$ com uma das $k-$colorações contadas em $\mu(d(v), k)$, podemos colorir os vértices de $V(G)\setminus N(v)$ com qualquer uma das $k$ cores disponíveis. Note que há $k^n$ formas de colorir $G$ com $k$ cores. Sendo assim: 
 	
 	\begin{align}
 		\begin{split}
 			\mathds{P}[X_v] = \dfrac{\mu(d(v), k) {\cdot} k^{n - d(v)}}{k^n} = \dfrac{\mu(d(v), k)}{k^{d(v)}} = \mathds{P}[Y_{d(v)}]
 		\end{split} 
 	\end{align}
 	
 \end{proof}
 
 
 \begin{lema}  
 	\label{lema3} 
 	$\mu(2d, k) \leq (2d - 1) \cdot k \cdot \mu(2d - 2, k)$, para $d \geq 1$.
 \end{lema}
 
 \begin{proof} Iremos construir as $k$-partições pares $\mathcal{P}$ possíveis de $2d$ elementos com base nas escolhas que temos para um determinado elemento $x \in [2d]$. Devemos escolher uma parte $P_i$ para $x$ pertencer e temos $k$ partes disponíveis para $x$. Como cada parte tem tamanho par, devemos escolher um elemento $y \in [2d]$ diferente de $x$ para pertencer também à parte $P_i$. Temos $2d - 1$ escolhas para este caso. Por fim, devemos particionar os $2d - 2$ elementos restantes em $k$ partes de modo que cada parte tenha tamanho par. Sendo assim, devemos escolher uma $k$-partição par $\mathcal{P'}$ de $2d - 2$ elementos. Logo, $\mu(2d, k) \leq (2d - 1) \cdot k \cdot \mu(2d - 2, k)$. 
 	
 \end{proof} \newl
 
 
  \begin{lema}  
 	\label{lema4} 
 	$\mu(2d, k) \leq (d {\cdot} k)^d$, para $d \geq 1$.
 \end{lema}
 
 \begin{proof} Iremos demonstrar por indução em $d$ que $\mu(2d, k) \leq 2d!! \cdot k^d$. Como $2d!! \leq d^d$, pela Proposição $\ref{prop3}$, disto segue que $\mu(2d, k) \leq (d {\cdot} k)^d$. 
 	
 	Base ($d = 1$): Pelo Lema $\ref{lema3}$, temos que $\mu(2, k) \leq k \cdot \mu(0, k)= 2!! \cdot k$ e o resultado segue.
 	
 	Passo ($d > 1$): Suponha que $\mu(2\ell, k) \leq (2\ell)!! \cdot k^{\ell}$, para $1 \leq \ell < d$. Pelo Lema $\ref{lema3}$, $\mu(2d, k) \leq (2d - 1) \cdot k \cdot \mu(2 {\cdot} (d - 1), k)$. Por $HI$, temos que: 
 	\begin{align}
 		\begin{split}
			\mu(2 {\cdot} (d - 1), k) \leq (2d - 2)!! \cdot k^{d-1}
 		\end{split} 
 	\end{align} 
 	
 	Portanto: 
 	\begin{align}
 		\begin{split}
 			\mu(2d, k) &\leq (2d - 1) \cdot k \cdot \mu(2 {\cdot} (d - 1), k) \\
 			&\leq (2d - 1) \cdot k \cdot (2d - 2)!! \cdot k^{d-1} \\
 			&\leq 2d!! \cdot k^d
 		\end{split} 
 	\end{align} 
 	
 \end{proof} \newl
 
 
\begin{lema}   
	\label{lema5}
	\begin{equation}
		\mu(2d, k) =
		\begin{cases}
			1 & \text{$k = 1$} \\
			\sum\limits_{i = 0}^{d} \escolhe{2d}{2i} {\cdot} \mu(2i, k-1) & \text{c.c.} \\ 
		\end{cases}
	\end{equation} 
\end{lema}
 
\begin{proof} 
	Se $k=1$, então $\mu(2d, k) = 1$, pois os $2d$ elementos devem estar contidos em uma única parte. Sendo assim, considere que $k > 1$. Agora, iremos construir as $k$-partições pares $\mathcal{P}=\{P_1, P_2, \ldots P_k\}$ possíveis de $2d$ elementos. Primeiro, devemos escolher quantos dos $2d$ elementos irão pertencer a parte $P_k$. Como $|P_k|$ é par, podemos escolher qualquer inteiro $i$ entre $0$ e $d$, de modo que $|P_k| = 2i$. Como os $2d$ elementos são distintos, temos $\escolhe{2d}{2i} = \escolhe{2d}{2d - 2i}$ maneiras de escolher $2i$ elementos para a parte $P_k$. Após isso, devemos particionar os $2d - 2i$ elementos restantes em $(k-1)$ partes de tamanho par, sendo assim, devemos escolher uma $(k-1)$-partição par $\mathcal{P'}$ de $2d - 2i$ elementos. Portanto: 
	
	\begin{align}
		\begin{split}
			\mu(2d, k) &= \sum\limits_{i = 0}^{d}\escolhe{2d}{2i} {\cdot} \mu(2d - 2i, k-1) \\
			&= \sum\limits_{i = 0}^{d}\escolhe{2d}{2d - 2i} {\cdot} \mu(2d - 2i, k-1)\\
			&= \sum\limits_{i = 0}^{d}\escolhe{2d}{2i} {\cdot} \mu(2i, k-1)  
		\end{split} 
	\end{align}  
\end{proof} \newl


\begin{lema}    
	\label{lema6}
	\begin{equation}
		\varphi_2(2d, k) =
		\begin{cases}
			0 & \text{se $k \leq 1$}\\
			2^{2d - 1} & \text{se $k = 2$} \\
			\sum\limits_{i = 1}^{d} \escolhe{2d}{2i} {\cdot} \varphi_2(2i, k-1) & \text{c. c.} \\ 
		\end{cases}
	\end{equation} 
\end{lema}

\begin{proof} 
	Iremos analisar cada caso da recorrência separadamente. \newl
	
	Caso 1 ($k \leq 1$): Se $k \leq 1$, então não há como $k$-particionar os $2d$ elementos de modo que apenas as partes $P_1$ e $P_2$ tenham tamanho ímpar. Portanto, $\varphi(2d, k) = 0$. \newl
	
	Caso 2 ($k = 2$): Se $k = 2$, então qualquer $k$-partição $\mathcal{P} = \{P_1, P_2\}$ de $2d$ elementos é contada em $\mu(2d, 2)$ ou em $\varphi_2(2d, 2)$, pois como temos um número par de elementos, ambas partes $P_1$ e $P_2$ têm tamanho par ou ímpar. Logo $\mu(2d, 2) + \varphi_2(2d, 2) = 2^{2d}$, pois $2^{2d}$ é o total de $2$-partições possíveis de $2d$ elementos. Pelo Lema $\ref{lema5}$:
	
	\begin{align}
		\begin{split}
			\mu(2d, 2) &= \sum\limits_{i = 0}^{d}\escolhe{2d}{2i} {\cdot} \mu(2i, 1) = \sum\limits_{i = 0}^{d}\escolhe{2d}{2i} = 2^{2d-1}
		\end{split} 
	\end{align} 
	
	Logo, temos que $\varphi_2(2d, 2)  = 2^{2d} - 2^{2d - 1} = 2^{2d - 1}$ e o resultado segue. \newl
	
	
	Caso 3 ($k > 2$): Iremos construir as $k$-partições não pares $\mathcal{P} = \{P_1, P_2, \ldots P_k\}$ possíveis de $2d$ elementos. Como $k > 2$, existe uma parte $P_i \in \mathcal{P}$, onde $|P_i|$ é par. Sendo assim, a demonstração segue de modo análogo à demonstração do Lema $\ref{lema5}$, com a única restrição de que $|P_i| < 2d$, pois as partes $P_1$ e $P_2$ tem ao menos um elemento.
\end{proof} \newl
  

\begin{lema} 
	\label{lema7} 
	$\mu(2d, k) \leq \mu(2d - 2, k) {\cdot} k^{2}$, para $d \geq 1$.
\end{lema}

\begin{proof}
	Iremos provar que $\mu(2d, k) = k {\cdot} \mu(2d - 2, k) + (k^2 - k) {\cdot} \varphi_2(2d - 2, k)$. Disto segue que $\mu(2d, k) \leq \mu(2d - 2, k) {\cdot} k^{2}$, pois, pelos Lemas $\ref{lema5}$ e $\ref{lema6}$, temos que $\varphi_2(2d - 2, k) \leq \mu(2d - 2, k)$, considerando que ambos possuem uma recorrência similar e ainda $\varphi_2(2d, 2) = \mu(2d, 2)$. Sejam dois elementos distintos $x, y \in [2d]$. Iremos construir uma $k$-partição par $\mathcal{P}$ de $2d$ elementos com base em duas escolhas: se $x$ e $y$ irão pertencer a mesma parte $P_i \in \mathcal{P}$ ou não. \newl
	
	Caso 1: Se escolhermos que $x$ e $y$ irão pertencer a mesma parte $P_i \in \mathcal{P}$, então devemos escolher uma parte $P_i$ das $k$ partes disponíveis. Após, devemos escolher uma $k$-partição par $\mathcal{P'}$ de $2d - 2$ elementos para os elementos restantes. Sendo assim, para este caso, temos $k {\cdot} \mu(2d - 2, k)$ partições possíveis. \newl 
	
	Caso 2: Se escolhermos que $x \in P_i$ e $y \in P_j$, onde $P_i \neq P_j$, então devemos escolher primeiro quais são as partes $P_i$ e $P_j$ dentre as $k$ partes que irão conter $x$ e $y$ respectivamente. Temos $2 {\cdot} \escolhe{k}{2} = k^2 - k$ formas de escolher $P_i$ e $P_j$, pois há $\escolhe{k}{2}$ maneiras de escolher duas das $k$ partes disponíveis e há duas maneiras de escolher qual das duas partes irá conter cada elemento. Após, devemos escolher uma $k$-partição não par $\mathcal{P'}$ onde apenas as partes $P_i$ e $P_j$ tenham tamanho ímpar. Por simetria das partes, há exatamente $\varphi_2(2d - 2, k)$ partições $\mathcal{P'}$ distintas. Logo, para este caso, temos $(k^2 - k) {\cdot} \varphi_2(2d - 2, k)$ partições possíveis.
	
	
\end{proof} \newl


\begin{lema}  
	\label{lema8} 
	Seja $Y_{2d}$ o evento de uma $k-$partição de $2d$ elementos $\mathcal{P}$ ser par, para $d \geq 1$. Seja $Y_{2d-2}$ o evento de uma $k-$partição de $2d - 2$ elementos $\mathcal{P'}$ ser par. Temos que $\mathds{P}[Y_{2d}] \leq \mathds{P}[Y_{2d-2}]$.
\end{lema}
 
\begin{proof} 
	Pelo Lema $\ref{lema7}$, temos que $\mu(2d, k) \leq \mu(2d - 2, k) {\cdot} k^{2}$. Logo:  
	\begin{align}
		\begin{split}
		  \mathds{P}[Y_{2d}] = \dfrac{\mu(2d, k)}{k^{2d}} \leq \dfrac{\mu(2d - 2, k) {\cdot} k^{2}}{k^{2d}} = \dfrac{\mu(2d - 2, k)}{k^{2d-2}} = \mathds{P}[Y_{2d-2}]
		\end{split} 
	\end{align}
	 
\end{proof}\newl 

 
\begin{proof}[Demonstração do Teorema~\ref{teo2}]
	Pinte os vértices de $G$ com $k = \ell {\cdot} \sqrt[\ell]{e {\cdot} \Delta^2}$ cores aleatoriamente e independentemente, onde $1 \leq \ell \leq \blceil \dfrac{\delta(G)}{2} \brceil$. Seja $X_{v}$ o evento de $v$ não ter testemunha ímpar, para $v \in V(G)$. Pelo Lema $\ref{lema2}$, temos que $\mathds{P}[X_v] = \mathds{P}[Y_{d(v)}]$, onde $Y_{d(v)}$ é o evento de uma $k-$partição de $d(v)$ elementos $\mathcal{P}$ ser par. Pelo Lema $\ref{lema8}$, temos que $\mathds{P}[X_v] = \mathds{P}[Y_{d(v)}] \leq \mathds{P}[Y_{2\ell}]$, para todo $v \in V(G)$. Pelo Lema $\ref{lema4}$,  $\mathds{P}[Y_{2\ell}] =\dfrac{\mu(2\ell, k)}{k^{2\ell}} \leq \dfrac{(\ell {\cdot} k)^\ell}{k^{2\ell}} = \left(\dfrac{\ell}{k}\right)^\ell$. Portanto: 
	
	\begin{align}
		\begin{split}
			\mathds{P}[X_{v}] \leq \mathds{P}[Y_{2\ell}] \leq \left(\dfrac{\ell}{k}\right)^\ell = \left(\dfrac{\ell}{\ell {\cdot} \sqrt[\ell]{e {\cdot} \Delta^2}}\right)^\ell = \dfrac{1}{e {\cdot} \Delta^2}
		\end{split} 
	\end{align}
	
	Note que cada evento $X_v$ é dependente a no máximo $\Delta^2 - \Delta$ outros eventos, para todo $v \in V(G)$. Sendo assim, pelo Lema Local de Lovász, temos que $\mathds{P}[\bigcap\limits_{v \in V(G)} \overline{X_{v}} ] > 0$. Portanto, existe uma $k$-coloração onde $v$ tem testemunha ímpar, para todo $v \in V(G)$. Logo, $\chi_{io}(G) \leq k = \ell {\cdot} \sqrt[\ell]{e {\cdot} \Delta^2}$, para $1 \leq \ell \leq \dfrac{\delta}{2}$.
	
\end{proof}\newpage
}
  
% PARTE TRÊS
% PARTE TRÊS
% PARTE TRÊS
% PARTE TRÊS

{
\begin{defi}
	Seja $G$ um grafo. Denotamos por $\tau(e)$ a quantidade de ciclos que a aresta $e \in E(G)$ pertence. Denotamos por $\tau(G) = max(\tau(e): \forall e \in E(G))$.
\end{defi} \newbegin

\begin{prop}
	\label{prop4}
	Se $G$ um grafo $k-$crítico, então $G$ é $(k-1)$-aresta-conexo.
\end{prop}\newbegin

\begin{teo}
	$\chi(G) \leq \tau(G) + 2$.
\end{teo}

\begin{proof}
	Suponha que o enunciado não vale e seja $G$ um contraexemplo com o menor número de arestas possível. Pela minimalidade de $G$, temos que $H$ não é um contraexemplo, para qualquer $H \subsetneq G$, \ie, $\chi(H) \leq \tau(H) + 2$. Como $\chi(G) \geq \tau(G) + 3$ e $\tau(G) \geq \tau(H)$, temos que:
	
	\begin{align}
		\begin{split}
			\chi(G) \geq \tau(G) + 3 &\geq \tau(H) + 3 \geq \chi(H) + 1 \\
			\therefore \chi(G) &> \chi(H)
		\end{split} 
	\end{align}
	
	Portanto, temos que $G$ é $\chi-$crítico, onde $\chi = \chi(G)$. Pela Proposição $\ref{prop4}$, temos que $G$ é $(\chi - 1)$-aresta-conexo. Como $G$ é $(\chi - 1)$-aresta-conexo, sabemos que $G$ possui pelo menos $\chi - 1$ $uv$-caminhos disjuntos nas arestas, para todo par distinto $u, v \in V(G)$. Sejam $P_1, P_2, \ldots P_{\chi - 1}$ caminhos disjuntos nas arestas de $G$. Seja $e$ a aresta que incide em $u$ em $P_1$. Note que $P_1 \cup P_j$ contém um ciclo $C$, tal que $e \in E(C)$, para $1 < j \leq \chi - 1$, pois $P_1$ e $P_j$ são $uv$-caminhos distintos. Logo $\tau(G) \geq \tau(e) \geq \chi(G) - 2$. Portanto, $\chi(G) \geq \tau(G) + 3 \geq \chi(G) - 2 + 3 = \chi(G) + 1$, uma contradição.
	
\end{proof}
}
 
\end{document}
 