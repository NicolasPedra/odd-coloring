 
 
\begin{defi}
	\label{defi1}
	Seja $\mathcal{P} = \{P_1, P_2, \ldots P_k \}$ uma $k-$partição de $d$ elementos distintos. Dizemos que $\mathcal{P}$ é uma $k-$partição par se $|P_i|$ é par, para todo $P_i \in \mathcal{P}$.
\end{defi} \newbegin
 

\begin{defi}
	\label{defi2}
	Denotamos por $\mu(d, k)$ a quantidade de $k-$partições pares distintas de $d$ elementos distintos.
\end{defi} \newbegin

\begin{defi}
	\label{def3}
	Denotamos por $\varphi_2(d, k)$ a quantidade de $k-$partições $\mathcal{P}$ não pares de $d$ elementos onde somente as duas primeiras partes $P_1, P_2 \in \mathcal{P}$ possuem cardinalidade ímpar. 
\end{defi} \newbegin

\begin{defi}
	\label{def4}
	Seja $\mathcal{H}$ um hipergrafo e seja $C : V(\mathcal{H}) \rightarrow [k]$ uma $k$-coloração dos vértices de $\mathcal{H}$. Dizemos que $C$ é uma $k$-coloração ímpar se, para toda aresta $e \in E(\mathcal{H})$, existe uma cor que aparece impar vezes nos vértices de $e$. Denotamos por $\chi_{io}(\mathcal{H})$ o menor inteiro $k$ tal que  $\mathcal{H}$ possui uma $k-$coloração ímpar.
	
\end{defi} \newbegin

\begin{defi}
	\label{def5}
	Seja $G$ um grafo e seja $C : V(\mathcal{H}) \rightarrow [k]$ uma $k$-coloração dos vértices de $G$. Dizemos que $C$ é uma $k$-coloração ímpar se, para todo vértice $v \in V(G)$, existe uma cor que aparece impar vezes na vizinhança de $G$. Denotamos por $\chi_{o}(\mathcal{H})$ (resp. $\chi_{io}(\mathcal{H})$) o menor inteiro $k$ tal que $G$ possui uma $k-$coloração ímpar própria (resp. não própria).
\end{defi} \newbegin


\begin{prop}
	\label{prop1}
	Seja $2n!!$ o fatorial dos ímpares. Temos que $2n!! \leq n^n$.
\end{prop} 
 