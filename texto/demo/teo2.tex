  \section{Demonstração do Teorema $\ref{teo2}$} \newl

\begin{proof}[Demonstração do Teorema~\ref{teo2}]
	Iremos provar que $\chi_{o}(G) \leq \chi(G) {\cdot} \chi_{io}(G)$. Disto segue que $\chi_{o}(G) \leq \chi(G) {\cdot} t {\cdot} (e {\cdot} (\Delta^2 + 1))^{1/t}$, pois, pelo, Corolário $\ref{cor2}$, temos que $ \chi_{io}(G) \leq t {\cdot} (e {\cdot} (\Delta^2 + 1))^{1/t}$. Seja $\mathcal{P} = \{P_1, P_2 \ldots, P_k\}$ uma $k$-coloração ímpar mínima. Para cada parte $P_i \in \mathcal{P}$, pinte os vértices de $P_i$ com uma coloração própria mínima de $G[P_i]$, utilizando cores distintas para cada parte $P_i \in \mathcal{P}$. Ao final, temos uma coloração $c$ de $G$ utilizando no máximo $\chi(G) {\cdot} \chi_{io}(G)$ cores. Afirmamos que $c$ é uma coloração ímpar e própria. Note que $c$ é própria, pois tome $uv \in E(G)$, se $u \in P_i$ e $v \in P_j$, tal que $i \neq j$ e $P_i, P_j \in \mathcal{P}$, então $u$ e $v$ possuem cores distintas, pela construção da coloração, se $u, v \in P_i$, então $u$ e $v$ possuem cores distintas, pois $P_i$ é colorido propriamente com $\chi(G[P_i])$ cores. Observe que, para todo $v \in V(G)$, existe uma parte $P_i \in \mathcal{P}$, tal que $|S|$ é ímpar, onde $S = P_i \cap N_G(v)$. Seja $T = N_G(v) \setminus S$. Note que existe uma cor em $c$ que aparece ímpar vezes em $S$, pois $|S|$ é ímpar. Como $c(s) \neq c(t)$, para $s \in S$ e $t \in T$, pois $s$ e $t$ pertencem a partes distintas de $\mathcal{P}$, temos que existe uma cor que aparece ímpar vezes em $N_G(v)$. 
	
\end{proof}