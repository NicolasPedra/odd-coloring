 \section{Demonstração do Teorema $\ref{teo1}$} \newl

\begin{lema}  
	\label{lema1} 
	$\mu(2d, k) \leq (2d - 1) \cdot k \cdot \mu(2d - 2, k)$, para $d \geq 1$.
\end{lema}

\begin{proof} Iremos construir as $k$-partições pares $\mathcal{P}$ possíveis de $2d$ elementos com base nas escolhas que temos para um determinado elemento $x \in [2d]$. Devemos escolher uma parte $P_i$ para $x$ pertencer e temos $k$ partes disponíveis para $x$. Como cada parte tem tamanho par, devemos escolher um elemento $y \in [2d]$ diferente de $x$ para pertencer também à parte $P_i$. Temos $2d - 1$ escolhas para este caso. Por fim, devemos particionar os $2d - 2$ elementos restantes em $k$ partes de modo que cada parte tenha tamanho par. Sendo assim, devemos escolher uma $k$-partição par $\mathcal{P'}$ de $2d - 2$ elementos. Logo, $\mu(2d, k) \leq (2d - 1) \cdot k \cdot \mu(2d - 2, k)$. 
	
\end{proof} \newl


\begin{lema}  
	\label{lema2} 
	$\mu(2d, k) \leq (d {\cdot} k)^d$, para $d \geq 1$.
\end{lema}

\begin{proof} Iremos demonstrar por indução em $d$ que $\mu(2d, k) \leq 2d!! \cdot k^d$. Como $2d!! \leq d^d$, pela Proposição $\ref{prop1}$, disto segue que $\mu(2d, k) \leq (d {\cdot} k)^d$. 
	
	Base ($d = 1$): Pelo Lema $\ref{lema1}$, temos que $\mu(2, k) \leq k \cdot \mu(0, k)= 2!! \cdot k$ e o resultado segue.
	
	Passo ($d > 1$): Suponha que $\mu(2\ell, k) \leq (2\ell)!! \cdot k^{\ell}$, para $1 \leq \ell < d$. Pelo Lema $\ref{lema1}$, $\mu(2d, k) \leq (2d - 1) \cdot k \cdot \mu(2 {\cdot} (d - 1), k)$. Por $HI$, temos que: 
	\begin{align}
		\begin{split}
			\mu(2 {\cdot} (d - 1), k) \leq (2d - 2)!! \cdot k^{d-1}
		\end{split} 
	\end{align} 
	
	Portanto: 
	\begin{align}
		\begin{split}
			\mu(2d, k) &\leq (2d - 1) \cdot k \cdot \mu(2 {\cdot} (d - 1), k) \\
			&\leq (2d - 1) \cdot k \cdot (2d - 2)!! \cdot k^{d-1} \\
			&\leq 2d!! \cdot k^d
		\end{split} 
	\end{align} 
	
\end{proof} \newl


\begin{lema}   
	\label{lema3}
	\begin{equation}
		\mu(2d, k) =
		\begin{cases}
			1 & \text{$k = 1$} \\
			\sum\limits_{i = 0}^{d} \escolhe{2d}{2i} {\cdot} \mu(2i, k-1) & \text{c.c.} \\ 
		\end{cases}
	\end{equation} 
\end{lema}

\begin{proof} 
	Se $k=1$, então $\mu(2d, k) = 1$, pois os $2d$ elementos devem estar contidos em uma única parte. Sendo assim, considere que $k > 1$. Agora, iremos construir as $k$-partições pares $\mathcal{P}=\{P_1, P_2, \ldots P_k\}$ possíveis de $2d$ elementos. Primeiro, devemos escolher quantos dos $2d$ elementos irão pertencer a parte $P_k$. Como $|P_k|$ é par, podemos escolher qualquer inteiro $i$ entre $0$ e $d$, de modo que $|P_k| = 2i$. Como os $2d$ elementos são distintos, temos $\escolhe{2d}{2i} = \escolhe{2d}{2d - 2i}$ maneiras de escolher $2i$ elementos para a parte $P_k$. Após isso, devemos particionar os $2d - 2i$ elementos restantes em $(k-1)$ partes de tamanho par, sendo assim, devemos escolher uma $(k-1)$-partição par $\mathcal{P'}$ de $2d - 2i$ elementos. Portanto: 
	
	\begin{align}
		\begin{split}
			\mu(2d, k) &= \sum\limits_{i = 0}^{d}\escolhe{2d}{2i} {\cdot} \mu(2d - 2i, k-1) \\
			&= \sum\limits_{i = 0}^{d}\escolhe{2d}{2d - 2i} {\cdot} \mu(2d - 2i, k-1)\\
			&= \sum\limits_{i = 0}^{d}\escolhe{2d}{2i} {\cdot} \mu(2i, k-1)  
		\end{split} 
	\end{align}  
\end{proof} \newl


\begin{lema}    
	\label{lema4}
	\begin{equation}
		\varphi_2(2d, k) =
		\begin{cases}
			0 & \text{se $k \leq 1$}\\
			2^{2d - 1} & \text{se $k = 2$} \\
			\sum\limits_{i = 1}^{d} \escolhe{2d}{2i} {\cdot} \varphi_2(2i, k-1) & \text{c. c.} \\ 
		\end{cases}
	\end{equation} 
\end{lema}

\begin{proof} 
	Iremos analisar cada caso da recorrência separadamente. \newl
	
	Caso 1 ($k \leq 1$): Se $k \leq 1$, então não há como $k$-particionar os $2d$ elementos de modo que apenas as partes $P_1$ e $P_2$ tenham tamanho ímpar. Portanto, $\varphi(2d, k) = 0$. \newl
	
	Caso 2 ($k = 2$): Se $k = 2$, então qualquer $k$-partição $\mathcal{P} = \{P_1, P_2\}$ de $2d$ elementos é contada em $\mu(2d, 2)$ ou em $\varphi_2(2d, 2)$, pois como temos um número par de elementos, ambas partes $P_1$ e $P_2$ têm tamanho par ou ímpar. Logo $\mu(2d, 2) + \varphi_2(2d, 2) = 2^{2d}$, pois $2^{2d}$ é o total de $2$-partições possíveis de $2d$ elementos. Pelo Lema $\ref{lema3}$:
	
	\begin{align}
		\begin{split}
			\mu(2d, 2) &= \sum\limits_{i = 0}^{d}\escolhe{2d}{2i} {\cdot} \mu(2i, 1) = \sum\limits_{i = 0}^{d}\escolhe{2d}{2i} = 2^{2d-1}
		\end{split} 
	\end{align} 
	
	Logo, temos que $\varphi_2(2d, 2)  = 2^{2d} - 2^{2d - 1} = 2^{2d - 1}$ e o resultado segue. \newl
	
	
	Caso 3 ($k > 2$): Iremos construir as $k$-partições não pares $\mathcal{P} = \{P_1, P_2, \ldots P_k\}$ possíveis de $2d$ elementos. Como $k > 2$, existe uma parte $P_i \in \mathcal{P}$, onde $|P_i|$ é par. Sendo assim, a demonstração segue de modo análogo à demonstração do Lema $\ref{lema3}$, com a única restrição de que $|P_i| < 2d$, pois as partes $P_1$ e $P_2$ tem ao menos um elemento.
\end{proof} \newl


\begin{lema} 
	\label{lema5} 
	$\mu(2d, k) \leq \mu(2d - 2, k) {\cdot} k^{2}$, para $d \geq 1$.
\end{lema}

\begin{proof}
	Iremos provar que $\mu(2d, k) = k {\cdot} \mu(2d - 2, k) + (k^2 - k) {\cdot} \varphi_2(2d - 2, k)$. Disto segue que $\mu(2d, k) \leq \mu(2d - 2, k) {\cdot} k^{2}$, pois, pelos Lemas $\ref{lema3}$ e $\ref{lema4}$, temos que $\varphi_2(2d - 2, k) \leq \mu(2d - 2, k)$, considerando que ambos possuem uma recorrência similar e ainda $\varphi_2(2d, 2) = \mu(2d, 2)$. Sejam dois elementos distintos $x, y \in [2d]$. Iremos construir uma $k$-partição par $\mathcal{P}$ de $2d$ elementos com base em duas escolhas: se $x$ e $y$ irão pertencer a mesma parte $P_i \in \mathcal{P}$ ou não. \newl
	
	Caso 1: Se escolhermos que $x$ e $y$ irão pertencer a mesma parte $P_i \in \mathcal{P}$, então devemos escolher uma parte $P_i$ das $k$ partes disponíveis. Após, devemos escolher uma $k$-partição par $\mathcal{P'}$ de $2d - 2$ elementos para os elementos restantes. Sendo assim, para este caso, temos $k {\cdot} \mu(2d - 2, k)$ partições possíveis. \newl 
	
	Caso 2: Se escolhermos que $x \in P_i$ e $y \in P_j$, onde $P_i \neq P_j$, então devemos escolher primeiro quais são as partes $P_i$ e $P_j$ dentre as $k$ partes que irão conter $x$ e $y$ respectivamente. Temos $2 {\cdot} \escolhe{k}{2} = k^2 - k$ formas de escolher $P_i$ e $P_j$, pois há $\escolhe{k}{2}$ maneiras de escolher duas das $k$ partes disponíveis e há duas maneiras de escolher qual das duas partes irá conter cada elemento. Após, devemos escolher uma $k$-partição não par $\mathcal{P'}$ onde apenas as partes $P_i$ e $P_j$ tenham tamanho ímpar. Por simetria das partes, há exatamente $\varphi_2(2d - 2, k)$ partições $\mathcal{P'}$ distintas. Logo, para este caso, temos $(k^2 - k) {\cdot} \varphi_2(2d - 2, k)$ partições possíveis.
	
	
\end{proof} \newl


\begin{proof}[Demonstração do Teorema~\ref{teo1}]
	Pinte os vértices de $\mathcal{H}$ com $k = t {\cdot} (e {\cdot} (\Gamma + 1))^{1/t}$ cores aleatoriamente e independentemente. Seja $X_{e}$ o evento da aresta $e \in E(\mathcal{H})$ não ter uma cor que apareça ímpar vezes, onde $|e|$ é ímpar. Note que $\mathds{P}[X_{e}] = \dfrac{\mu(|e|, k)}{k^{|e|}}$. Como $|e| \geq 2t$, pelo Lema $\ref{lema5}$, temos que: 
	\begin{align}
		\begin{split}
			\mu(|e|, k) \leq \mu(|e| - (|e| - 2t), k) {\cdot} k^{|e| - 2t} = \mu(2t, k) {\cdot} k^{|e| - 2t}
		\end{split} 
	\end{align} 
	
	Pelo Lema $\ref{lema2}$:
	\begin{align}
		\begin{split}
			\mu(2t, k) \leq (t {\cdot } k)^t
		\end{split} 
	\end{align} 
	
	Portanto, por $(7)$ e $(8)$:
	
	\begin{align}
		\begin{split}
			\mathds{P}[X_{e}] = \dfrac{\mu(|e|, k)}{k^{|e|}} 
			\leq \left(\dfrac{t}{k}\right)^t  = \left(\dfrac{t}{t {\cdot}   (e {\cdot} (\Gamma + 1))^{1/t}}\right)^t = \dfrac{1}{e {\cdot} (\Gamma + 1)}
		\end{split} 
	\end{align}
	
	Sendo assim, pelo Lema Local de Lovász, temos que $\mathds{P}[\bigcap\limits_{e \in E(\mathcal{H})} \overline{X_{e}} ] > 0$. Portanto, existe uma $k$-coloração tal que existe uma cor que aparece ímpar vezes nos vértices de $e$, para toda aresta $e \in E(\mathcal{H})$. Logo, $\chi_{io}(\mathcal{H}) \leq k = t {\cdot} (e {\cdot} (\Gamma + 1))^{1/t}$.
	
\end{proof}\newbegin